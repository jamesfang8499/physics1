\chapter{动量}

\section{教学要求}

动量是力学中的重要概念,动量守恒定律是自然界最重
要的普遍规律之一。因此在甲种本中单设一章来对有关知识
作较深入的讨论。

这一章的教学要求是:
\begin{enumerate}
\item 理解冲量和动量两个概念,掌握动量定理并会用它解
释一些物理现象。
\item 掌握动量守恒定律并会用它分析、计算有关的问题.
\item 理解弹性碰撞和非弹性碰撞,会计算一维弹性碰撞的
有关问题。
\item 了解反冲运动的原理及其在火箭技术中的应用。
\end{enumerate}

下面对这一章的教学内容作些具体说明。

冲量和动量是两个不容易理解的概念。教材在分析具体
事例的基础上,引用公式进行讨论,得出这两个概念,这样做,
对初学者来说,概念的物理意义可能清楚些,容易理解些。动
量的矢量性在研究动量定理和动量守恒定律时都很重要。但
是学生初学时往往对此认识不够,需要一开始讲解动量时,就
强调它的矢量性。

第二节讲解动量定理。要使学生明确:动量定理$Ft=
mv'-mv$表示的是力在一段时间内连续作用的累积效果与物
体动量变化间的关系;在变力的情况下,公式中的$F$表示变
力在时间$t$内的平均值。

动量守恒定律也可以用牛顿第二定律和第三定律推导出
来。这种讲法的好处是比较简便。然而,动量守恒定律是一
个独立的实验定律,而且它的适用范围比牛顿运动定律更广
泛。因此,教材以实验为基础总结出动量守恒定律,然后说明
这一定律与牛顿运动定律的一致性。

第三节讲述相互作用的物体的动量变化的实验,是为第
四节讲动量守恒定律作准备的。在归纳出动量守恒定律时应
向学生说明,这个定律是在分析研究大量实验事实的基础上
建立起来的,而不是仅靠几个实验得出来的。还应使学生清
楚系统动量守恒的条件,防止解题时不问条件,乱套公式的
倾向。

第五节讲解动量守恒定律和牛顿运动定律的关系。应当
使学生清楚地了解这两个定律在牛顿运动定律运用的范围内
是一致的,这两个定律的区别在于适用范围的不同。牛顿运
动定律只适用于宏观物体的低速运动情况,而动量守恒定律
的适用范围更普遍,不论是宏观物体还是微观粒子,低速运动
还是高速运动,相互作用是什么性质的,都遵守动量守恒
定律。

第六节讲述如何运用动量守恒和动能守恒来研究碰撞问
题。应该使学生明确,对于弹性碰撞需要同时运用动量守恒
和动能守恒来解决。讲解弹性碰撞,要注意培养学生运用两
个守恒定律分析解决问题的能力,进一步认识守恒定律的重
要性,例题中得出的两个钢球发生弹性正碰后的速度表达式,
应要求学生会自行推导,而不是记住它们。同时,要求学生能
够利用表达式讨论一些具体情况.例如,$m_1>m_2$, $m_1=m_2$,
$m_1<m_2$时,两钢球碰撞后的运动情况;$m_1\ll m_2$, $m_1=m_2$, 
$m_1\gg m_2$时,两钢球碰撞中能量的传递情况.并能根据讨论的
结果,解释一些弹性碰撞中发生的现象。

第七节反冲运动中讲述的内容都属于常识性的介绍,目
的是使学生了解物理知识在现代科学技术中的应用,介绍一
些有关的科学技术常识,开阔学生的眼界。

\section{教学建议}
全章教学可分三个单元。第一单元(第一、二节)讲授冲
量和动量的概念以及动量定理。第二单元(第三至五节)讲授
动量守恒定律,这是本章的重点。第三单元(第六、七节)讲
授碰撞和反冲运动,这是动量守恒定律的应用。

\subsection{第一单元}
\subsubsection{冲量和动量概念的引入}

冲量和动量这两个概念也
同功和能一样,不是一引入就能体会它的物理意义的。对这
两个概念的理解,要通过整章的教学过程逐步加深。所以第
一节的教学既要使学生初步了解冲量和动量的意义,特别是
为什么要引入这两个概念以及两个物理量的定义、计算式、单
位和矢量性等,又不能操之过急,要求过高。

课本在引入冲量和动量的概念时,通过日常生活中常见
的事实,以开动汽车为例,说明汽车获得一定的速度不仅同它
的牵引力有关,而且同力的作用时间有关。然后应用牛顿运
动定律和运动学的知识,得出了速度的变化跟力和作用时间
的关系$Ft=mv$. 再通过对这个式子的讨论给出冲量和动量
的定义。这样引入,除了物理意义比较明显、较易理解外,还
可以使学生从一开始就认识这两个概念之间的密切联系,了
解冲量(它表现力的作用)的效果是使物体获得动量。

教学中还应该使学生了解速度和动量的联系和区别。速
度和动量都可以作为运动的量变。但是速度只告诉我们物体
运动的慢快和方向,没有告诉我们使物体运动或者停止运动
需要多大的冲量。动量没有告诉我们物体运动速率的大小,
却能告诉我们使物体运动或停止运动所需的冲量。所以速度
是一个运动学的量,只能用来描述运动,而动量则是一个动力
学的量,它跟冲量即物体运动变化的原因相联系。这里,可以
用课本上以相同的速度运动的铅球和乒乓球的例子来说明。

\subsubsection{动量的变化}

学生初学动量时,往往忽略动量的矢量
性,只注意它的大小,不注意它的方向,以为只要物体的速度
大小不变,动量也就不改变.教学时可利用课本252页的例
题纠正这种错误认识。要想学生考虑钢球从坚硬的障碍物上
弹回,动量的大小并没有变,动量的变化为什么不等于零?原
因是动量是矢量,它的方向改变了,所以动量发生了变化。

教材明确指出,所谓动量的变化就是变化后的动量减去
变化前的动量,因此这里所说的动量的变化实际上就是上一
章所说“增加”或“增量”。具体计算时,由于我们只研究一
直线上的动量变化,所以只要先选定一个方向为正,就可以
把矢量运算简化为代数运算。

\subsubsection{关于打击时的平均作用力}
动量定理可以看作是牛顿第二定律的变形,在这里没有
必要作过多的讲解,可只限于用来解释一些这个定理便于解
释的现象,如为什么茶杯掉在地上要碎,而掉在软的东西上就
不易摔碎等。在计算方面,只限于计算打击、碰撞等问题中的
平均作用力,如课本255页中的例题那样。

课本255页例题后面有一段讨论.讨论中说“如果把铁
锤的重量也考虑在内,那么,这时道钉所受的打击是上面算
出的打击力加上铁锤的重量。”如果学生对此提出怀疑,教学
中可加以说明。应该指出,动量定理$Ft=mv'-mv$中的$F$是
物体所受的合力。所以在这个例题中应该列出如下方程:
\[F=\frac{mv'-mv}{t}=2.5\x10^3{\rm N}\]
而$F=N-G$, 所以
\[N=F+G=2.5\x10^3+49=2549{\rm N}\]

至于什么时候可以忽略铁锤的重量$G$, 这要看$G$与$F$的
差别有多大.象书上例题中$G$为$F$的2\%, 就可以忽略了。

\subsection{第二单元}
本单元是全章的重点。整个单元就是围绕一个中心——
动量守恒定律展开的。

\subsubsection{相互作用物体的动量变化}

第三节教学的关键是要
做好研究相互作用物体的动量变化的演示实验。实验可分为
两部分.第一部分为定性研究,如课本图8.2那样,目的是说
明两个相互作用的物体动量会发生变化,在课堂上可用两个
压紧弹簧的小车代替图8.2中的两个人,也可以用玩具小车
在可动的平板上的运动来演示。

第二部分为定量研究,要求分析课本插页的气垫导轨上
相互作用滑块的闪光照片,得出动量变化总是大小相等方向
相反的结论,教学中要注意引导学生学会如何根据闪光照片
计算滑块速度,并测量动量变化的方法,培养分析原始实验资
料的能力。根据学校的实验设备条件,也可以在课堂上当场
演示,得出同样结论。例如用光电计时器对气垫导轨上滑块
的速度测定来研究共动量的变化;用打点计时器研究相互作
用小车的动量变化;利用平抛原理测定相互作用小球的动量
变化等,有条件的学校还可以让学生自己动手来得出这个结
论。这一节的实验做得越好,对下一节动量守恒定律的学习
就更有利,教学中要十分重视。

\subsubsection{动量守恒的语言表达}

从“相互作用的物体的动量变
化总是大小相等、方向相反”的结论到“系统的动量守恒”的推
理过程,对语言表达的要求较高,教学中教师除应注意自己
授课语言的准确外,还应注意培养学生学习物理的语言表达
能力。推理过程有三个层次:第一是实验结论,“相互作用的
物体的动量变化总是大小相等,方向相反的”;第二是引入系
统总动量的概念,得出“系统的总动量的变化为零”;第三,再
得到系统的总动量守恒,以上三个层次,用公式表达,可与语
言表达对照如下(表中的$p_1$、$p'_1$、$p_2$、$p_2'$都是矢量。如果两个
物体作用前后都在一直线上运动,则上述四个量均代表带正
负号的动量数值。其中最后一式常常写成$m_1v_1+m_2v_2=m_1v'_1
+m_2v_2'$)。

\begin{center}
\begin{tabular}{p{.35\textwidth}c}
    \hline
    用语言表达&用公式表达\\
    \hline
    相互作用物体的动量变化总是大小相等,方向相反的&$p'_1-p_1=-(p'_2-p_2)$\\
    总动量的变化为零&$(p'_1+p'_2)-(p_1+p_2)=0$\\
    系统的总动量守恒&$p'_1+p'_2=p_1+p_2$\\
    \hline
\end{tabular}
\end{center}

\subsubsection{动量守恒定律的条件}
课本明确指出,系统动量守恒
的条件是“系统不受外力或所受外力的合力为零”。教学中要
重视培养学生在应用动量守恒定律时先检验是否符合守恒定
律条件的习惯,防止随意乱套公式,但是有一点应该向学
说明,在有些问题中,系统虽然受到外力作用,而且合力不为
零,但合外力同内力相比非常小,可以忽略不计时,动量守恒
定律仍然可以适用.例如课本260页例题,在列车间相互作
用时,重力和轨道支持力虽然合力为零,但列车和轨道间总有
摩擦力存在。由于这个摩擦力同列车间的内力相比很小,所
以仍可用动量守恒定律。又如手榴弹爆炸时,虽然整个系统
受到重力作用,但比起炸药的爆炸力来重力很小,可以忽略,
因此还是可以用动量守恒定律来计算爆炸后碎块的速度。

\subsubsection{动量守恒定律是普遍适用的物理规律}

动量守恒定
律是自然界最重要的普遍规律之一。教材在第三节末说了
相互作用物体的动量变化总是大小相等方向相反的结论不仅
适用于正碰,而且适用于斜碰。在第四节第一段,教材又说到
这个结论“在任何情况下”都成立,在第四节中,又特别提出三
点说明,对上述“任何情况”作了具体的解释。为了使学生对动
量守恒定律的普遍性有深切的了解,教学中可以多举一些实
例,特别是对说明中第1点可多举例说明,例如,子弹射入木
块、火车车厢连接在一起,课本插页图8.6的碰撞,属于粘合
在一起的例子,炮弹和手榴弹的爆炸属于分裂成碎块的例子。
此外,枪炮发射弹丸,人在船上行走等相互作用的例子,也可
用动量守恒定律来研究.对第2、3点,由于学生知识水平
有限,不可能理解得很具体,只要有一个印象就可以了。

\subsubsection{动量守恒定律中物体的速度}

在两个物体相互作用
对应用动量守恒定律可用表达式
\[m_1v_1+m_2v_2=m_1v'_1+m_2v'_2\]
式中涉及四个速度,要向学生指出,这四个速度必须是相对于
同一参照系的,一般都以地面为参照系。教师在引导学生解决
如人在船上行走之类的问题时,也要注意不要涉及相对速度,
而应把问题局限在相对于同一参照系研究其动量守恒。

\subsubsection{寻找“守恒量”的一个例子}

教材在第四节后安排了
一段阅读材料,要引导学生认真阅读,使学生了解到十六、七
世纪的哲学家如何从观察宇宙间各种物质的不断运动得出宇
宙运动的总量不会减少的看法,笛卡儿和牛顿又怎样去努力
寻找一个物理量来量度这种永恒的运动。把这一段阅读材料
同上一章第十节所说的寻求“守恒量”的重要意义联系起来
(课本244页),可以使学生体会到守恒定律在物理学里的重
要地位。这一段阅读材料中关于笛卡儿寻找量度运动的合适
物理量时的失误和对他的功绩的评述对学生也是有启发的。

\subsubsection{动量守恒定律和牛顿运动定律的关系}

教材第五节
主要说明两个问题,
\begin{enumerate}
 \item 动量守恒定律和牛顿运动定律是一致
的.但现在已认识到,动量守恒定律具有更大的普遍性;
\item 由于动量守恒定律不涉及相互作用的中间过程,所以在处理
某些问题时会更简便。
\end{enumerate}

教学中为了说明这两点,除了按教材讲述外,还可以选择
一道不要求研究中间过程的例题(例如课本260页的例2, 用
动量守恒定律和牛顿运动定律两种方法来解,说明两者的一
致性,同时又可以比较用动量守恒定律解题得更简便一些。

\subsection{第三单元}
\subsubsection{寻找“守恒量”的又一例子}

教材第六节从两个质量
相等的小球作弹性碰撞的演示开始,讨论只有动量守恒定律
还不能解释为什么现象是唯一的。最后得出在这种碰撞中还
应满足动能守恒。这是寻找“守恒量”的又一个例子。通过学
习,学生可以体会到,在一个物理过程中多一个“守恒量”,就
多一个制约因素,只有存在足够的制约因素,现象才能是唯
一的。在哪些现象中有什么守恒量,这就是自然界的规律,学
习物理学就要掌握这些规律。因此寻找“守恒量”是有重要意
义的。这一段教学内容很有启发性,对培养学生分析问题的
能力很有帮助.教学中首先要做好课本图8.7所示的演示实
验。在分析这一实验时,要使学生弄清初末状态的情况,所谓
碰撞的初末状态,指的是两球接触的短暂时间的前后,两
个状态的小球位置都是在最低点,而不是在弹起的最高点。
虽然初末状态的小球位置一样,但它们的速度不一样。在定
义了弹性碰撞以后,教材接着介绍了非弹性碰撞和完全非弹
性碰撞。教师也可在堂上利用课本图8.7的装置演示一下完
全非弹性碰撞.用课本图8.7演示弹性碰撞和完全非弹性碰
撞只能是定性的。有条件的学校可以用气垫导轨和光电计时
器演示,这样就可以作定量的测量,来验证弹性碰撞中动能守
恒,而非弹性碰撞中动能有损失。

\subsubsection{弹性碰撞末速度公式的推导}

课本第六节例题是根
据动量守恒和动能守恒列出方程组解弹性碰撞问题的一个例
子。例题中推导了碰撞后两钢球的速度,教学中要对推导方
法作一示范,並要求学生能自行推导,而不要简单地记忆末
速度公式。对例题后面的讨论,教学中也要加以重视。在讨
论中,可以将练习四的第4题一起进行讨论.从方程解中讨
论几种情况的物理意义,是一种重要的能力。对于弹性碰撞
的讨论是培养这方面能力的好时机。

对于弹性碰撞问题,课本局限在讨论正碰,并且两球中有
一球原来是静止的。对于学有余力的同学,两球初速均不为
零的弹性正碰可以作为练习题让他们自己推导和讨论,只要
他们掌握了推导方法,这样做是不太困难的。至于斜碰则不
要让学生去做了。

\subsubsection{反冲运动的方程}

反冲运动问题可以用动量守恒定
律来解决,如果反冲运动发生前物体是静止的,则动量守恒定
律可写成$MV+mv=0$的形式,其中$M$、$m$分别为向两个方向
运动的物体的质量,$V$、$v$是相应的速度,其正负号由假设的正
方向决定,可以让同学根据动量守恒定律自己写出这一方程。
但同学在写这一方程时,常常有两个容易出错的地方。第一,
如果$M$表示反冲运动发生前的总质量,则方程应改写为
$(M-m)V+mv=0$. 第二,有的同学根据反冲的两部分动量
大小相等写出$MV=mv$, 算出答案往往也是对的。但此式中
$V$和$v$均为绝对值。这种写法与本章前面的做法不一致,要
特别小心。如果反冲系统原来的动量不为零,则不能用这个
办法了。

\subsubsection{关于火箭的教学}

第七节的主要篇幅是介绍反冲运
动在火箭技术中的应用,主要是为了扩大知识面。教材不要
求作定量计算,所以关于火箭的最终速度与喷气速度、质量比
的关系不必介绍计算公式。至于火箭技术的一些原理、应用
及我国火箭技术发展情况,可鼓励学生自己查阅科普小册子
和杂志中的有关资料。这既可以激发学生的学习兴趣,又可
以培养学生自己查阅资料的能力。

















































































































































































































































































































































































\section{参考资料}
\subsection{机械运动中动量及动能的区别}

课本里的阅读材料:《笛卡儿和动量守恒定律》中已经提
到动量这个概念是笛卡尔、牛顿先后提出的,并且笛卡儿明确
地把物体的质量和速度的乘积作为物体“运动量”的量度。在
历史上由于他们的影响,在十七世纪四十年代至八十年代,科
学界普遍承认$mv$是机械运动唯一的量度。

在这同一时期内,由于惠更斯对完全弹性碰撞的研究,得
出了“各个质量和各个速度的平方乘积之和,在碰撞前后不
变”的结论。

1686年德国数学家莱布尼兹通过对落体运动的分析,认
为物体的质量和速度平方的乘积——活力——才是机械运动
的真正量度,从而与笛卡尔的主张展开了争论。

关于两种运动量度的争论,持续了近二百年,许多著名的
数理学家参加到争论中。后来随着力学本身的发展,人们对
这两种量度取得了清楚的认识。

牛顿第二定律
\[\frac{\dd(mv)}{\dd t}=F\]
所表现的只是运动的原因(力)
和结果(动量变化)之间的瞬时关系。如果考察力在一段时间
内的累积效应,可由上式得出:
\[\dd(mv)=F\cdot \dd t\]
\[\therefore\quad mv_2-mv_1=\int^{t_2}_{t_1}F\cdot \dd t\]
这就是动量定理:在一段时间内物体动量的变化,等于物体在
同一时间内所受外力的冲量,如果要根据物体在力的作用下
所通过的距离来考察力的作用效果,即力的空间积累效应,则
可得出:
\[\begin{split}
    F\cdot \dd s&=\frac{\dd(mv)}{\dd t}\cdot \dd s\\
    \int^{t_2}_{t_1}F\cdot \dd s&=\int^{t_2}_{t_1}v\cdot \dd (mv)
\end{split}\]
\[\therefore\quad \frac{1}{2}mv^2_2-\frac{1}{2}mv^2_1=\int^{t_2}_{t_1}F\cdot \dd s\]
这就是动能定理:物体动能的增加,等于外力对物体所做
的功。

所以,在力学中动量的变化表现着力的时间累积效应,动
量的变化与外力的冲量相等;动能的变化表现着力的空间累
积效应,动能的变化与外力的功相等。动量是与冲量密切联
系着的,动量决定物体反抗阻力能够移动多么久;动能是与
功密切联系着的,动能决定物体反抗阻力能够移动多么远。


\subsection{动量守恒定律的适用范围比牛顿运动定律广}
动量守恒定律比牛顿运动定律的适用范围要广。近代的
科学实验和理论分析都表明:在自然界中,大到天体的相互作
用,小到质子、中子等基本粒子间的相互作用都遵守动量守恒
定律。

在天文学中发现过这样一种现象:在太空的某个地方有
时会突然发出非常明亮的光,这就是超新星,可是它很快就
暗淡下来,经过几十个昼夜亮度就会减弱一半,光要从这样
一颗超新星出发到达地球需要几百万年,而相比之下超新星
从发光到熄灭的时间就显得太短了,在光到达我们这里以前,
超新星早已烧光了。

当光从超新星到达地球时,它给地球一个轻微的推动,而
与此同时地球却无法给超新星一个轻微的推动,因为它已消
失了,因此,如果我们想象一下超新星与地球之间的相互作
用力,在同一瞬间也就不是什么大小相等,方向相反了。此时,
牛顿第三定律显然已不适用了。

虽然如此,动量守恒定律还是正确的。不过,我们必须把
光也考虑在内。当超新星发射光时,星体反冲,得到动量,同
时光也带走了大小相等、方向相反的动量。经过几百万年光
到达地球时,光把它的动量给了地球。这里要注意的是:动量
不仅可以为实物所携带,而且可以以辐射的方式传递动量,当
我们考虑到这点时,动量守恒定律还是正确的。

\subsection{相对论的动量}

在牛顿力学里,动量定义为$mv$, 质量$m$是个不变的量。根
据牛顿第二定律,一个恒定的力,持续作用于一个物体,可以
使该物体有任意大的高速度。但是在现实中,真空中的光速是
极限速度,并且在任何条件下物体的速度都不可能超过真空
中的光速。因此,在高速运动时,认为质量,以及动量,是与速
度无关的,是不正确的。

相对论告诉我们,在高速运动时质量不再是一个不变的
量,而是随着运动的速度接近光的速度$c$而增大,如果用$m_0$
表示静止物体的质量,则以速度$v$运动的物体的质量$m$可以
用下式表示:
\[m=\frac{m_0}{\sqrt{1-\dfrac{v^2}{c^2}}}\]
相对论的动量仍定义为,
\[p=mv=\frac{m_0v}{\sqrt{1-\dfrac{v^2}{c^2}}}\]
在采用这样定义的情况下,牛顿本人所用的第二定律的表
达式
\[\frac{\dd p}{\dd t}=F,\qquad F=\frac{\dd}{\dd t}\cdot\frac{m_0v}{\sqrt{1-\dfrac{v^2}{c^2}}}\]
在接近光速的情况下也同样适用了,因为随着运动速度的增
大,决定物体惯性大小的质量也增大。当$v\to c$时,$m\to \infty$, 所
以加速度趋于零,不论力作用多长时间,速度也不会超过
光速。

对于静止质量$m_0=0$, 而速度为$c$的光子来说,它的动量$p=E/c$
可以这样推得:

因为$E=mc^2$,所以
\[\begin{split}
    E^2=m^2c^4&=m^2c^4-m^2v^2c^2+m^2v^2c^2\\
    &=m^2c^4\left(1-\frac{v^2}{c^2}\right)+p^2c^2=m^2_0c^4+p^2c^2
\end{split}\]
在$v\to c$, $m_0\to 0$时,$E^2=p^2c^2$, 所以
\[p=\frac{E}{c}\]


