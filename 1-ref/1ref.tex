\chapter{力}
\section{教学要求}
这一章讲述力的初步知识,为了减少学生开始学习高中
物理时遇到的困难,降低与初中物理的台阶,本章不讲静力学
的知识,只讲学习动力学所必需的预备知识。


这一章的教学要求是:

\begin{enumerate}
\item 正确理解力的概念,认识力是物体对物体的作用;知
道重力、弹力、摩擦力的产生条件以及它们的大小和方向;
掌握倔强系数,会计算弹簧的弹力;掌握滑动摩擦系数,会计
滑动摩擦力。
\item 进一步认识力是物体间的相互作用,掌提牛顿第三
定律。
\item 
初步学会分析物体的受力情况,会画物体受力图。
\item 理解合力和分力的概念,掌握平行四边形法则,知道
三角形法,会用作图法和公式法求合力和分力。
\end{enumerate}

在重力的教学中讲到了物体静止时拉紧悬绳的力或压在
水平支持物上的力,其大小等于物体所受的重力。在这节很
难从道理上把这一点讲清楚,先要求学生作为事实接受下来,
讲过牛顿第三定律后再来解决,这里所以要提到这一点,是
使学生对重力的大小有个具体认识,但主要还是因为在讲牛
顿第三定律之前就要用到它,如在计算摩擦力的习题中,在
测定滑动摩擦系数的学生实验中,都要用到.

弹力的方向问题比较复杂,在中学阶段经常遇到的弹力
大都是支持力和拉力,因此,弹力的方向是就支持力和拉力这
两种情形来讲解的,并要求学生掌握,以便以后进行力的分析。

在讲述胡克定律之前,为了扩展学生的眼界,先就几种基
本的形变定性说明:形变越大,弹力也越大,对金属丝的扭转
形变,说明扭转角度越大,弹力也越大,是考虑到以后讲卡文
迪许扭秤和库仑扭秤时要用到,这里并不要求详细地讨论。胡
克定律是就弹簧的弹力来讲的,给出了倔强系数的概念,但不
要求全面分析它跟弹簧的哪些因素有关,胡克定律的公式写
成$f=kx$, 没有写成$f=-kx$, 是考虑到这里还没有讲一维矢
量的运算,写成前者初学者容易接受;讲到简谐振动时再考虑
弹力的方向,写成后者。

摩擦力的教学,重点是讲述滑动摩擦力,静摩擦的教学,
只要求学生了解静摩擦和最大静擦力的概念,不讲静摩擦系
数。在讲解滑动摩擦时提到相对滑动,在讲解静摩擦时提到
相对运动趋势,只是为了讲解得确切一些,不要求涉及较为复
杂的情形(例如一个物体在传送带上相对滑动或相对静止)来
展开讲解。

讲述物体受力分析,既是教给学生一种分析方法,也是前
面学过的知识的综合运用。考虑到学会物体受力分析要贯穿
在整个力学教学中,本章只限于分析最基本的事例。讲述分
析方法,要强调明确研究的对象,分析时应强调力是物体对物
体的作用。至于按照什么顺序(如按重力、弹力、摩擦力的顺
序)来分析力,不宜过分强调,强调得过分,甚至要求学生死
记住一个分析的顺序,对于学生灵活运用知识,学会分析方
法,都是没有好处的。物体的受力情况实际上往往是很复杂
的,根据具体的问题,可以略去某些次要因素。这种研究问题
的方法,应该作为一项要求向学生提出,在以后作力的分析时
也要注意这一点。

力的合成和分解的教学,主要是使学生掌握力的平行四
边形法则。三角形法在实际中常常用到,学生应当知道。但
应该使学生明确,三角形法并不是另外一种新方法,只是平行
四边形法的简化。


\section{教学建议}
这一章内容,建议在教学中分为三个单元:

第一单元(第1节——第7节)讲述力的概念和力学中常见
的三种力,并在此基础上讲述牛顿第三定律和物体受力情况
分析。

第二单元(第8节——第10节)讲述力的合成和分解.

第三单元(第11节——第12节)在前面讲过的力的矢量性
的基础上,讲述矢量的初步知识。

\subsection{第一单元}
这一单元,首先通过重力、弹力、摩擦力这三种常见的力
和牛顿第三定律的教学,使学生比较具体地认识力的概念和
性质,包括力是物体之间的相互作用,力有大小、方向和作用
点等,然后引导学生初步掌握物体受力情况分析的方法,为进
一步学习力学知识打下基础。

\textbf{1.力的概念和图示}
\quad 
这部分内容大部分在初中已经学
过,是复习性的。但是有一些学生,虽然能记住学过的知识,
并不真正理解、会用,所以对这部分内容的教学仍然要给予
足够的重视。

力的概念的教学,最主要的是通过演示和说明使学生真
正理解力是物体对物体的作用,只要有力就一定有施力物体
和受力物体,力不能离开物体而存在。这是力学中最基本的
事实。可以通过一些实际的例子让学生指出施力物体和受力
物体,学生牢牢地掌握这一点,就不会离开物体的作用凭空设
想出多余的力来,有助于学好后面的物体受力分析。

关于力的图示,要强调表示力的每条有向线段,都要根据
选择好的标度,按照一定的比例来画。有的同学,往往用不同
的标度来画同一物体所受的不同的力,这种错误应该及时给
予纠正,一般情况下,力的作用点都可画在物体的质心上。由
于教材没有介绍质心这个概念,在实际作图时,只要在表示物
体的图形中间选择一个适当的点来表示力的作用点就可以
了,但是也有个别情况,例如第一次介绍摩擦力的方向时(课
本图1.12),为了清楚起见,把摩擦力$f$画在接触面上,而且
还把相互接触的两个物体画得离开一些。以后还应逐步使学
生知道,在分析力学问题时,有时只须画出力的示意图。力的
示意图常常是为了使物体的受力分析更清楚而作的,它在力
的大小、标度上的要求,不象力的图示中要求的那么严格,但
是对力的方向、力的相对大小也不能画错。在另一些情况下,
例如,在物体的受力分析之后,需作它的受力图并用图示法来
解题,这时的受力图又必须按力的图示的要求来作图了。但
是在本节的教学中,只须按课本上的要求进行就可以了。

\textbf{2.重力}\quad 关于重力的概念,按教材上所讲的“由于地球
的吸引而使物体受到的力叫做重力”来进行教学是适宜的,它
既浅显易懂,也没有把引力与重力等同起来。这里也不宜过
早的把地球对物体的引力与重力相区别,以后在教学中会逐
步讲清楚的。

应该注意的是,教材上把重量定义为物体所受的重力,按
照国务院颁布的法定计量单位,重量是质量的同义词。这种
差别的产生是因为教材是在法定计量单位颁布前编写的。这
一点,在教学中应该向同学们讲清楚。

物体所受重力的大小,可以用静力学的方法来确定,这就
是教材上讲的物体静止时拉紧悬绳的力或压在水平支持物上
的力,它们的大小都跟物体所受的重力相等。应该让学生注
意的是,重力是地球作用在物体上的力,受力者是物体,而物
体对悬绳的拉力或对支持物的压力,受力者是悬绳或支持物,
它们跟重力是作用在不同物体上的力,不能把它们跟重力相
混淆。

重力这一节的教学,也是具体地通过这种常见的力来表
明:力是物体对物体的作用,力有大小、方向和作用点,从而使
学生逐步加深对力这个抽象概念的理解。重心的概念,也是
从重力的概念引伸出来的,因为课本不讲同向平行力的合成,
不要追求概念的严谨,补充重心的定义,通过课本中图1.2
的演示,很容易说明“一个物体的各部分都要受到地球对它的
作用力,我们可以认为重力的作用集中于一点,这一点叫做物
体的重心。”

\textbf{3.形变与弹力}\quad
这一内容的教学,应着重于有关形变与
弹力的实验演示(包括显示微小形变的实验演示),使学生从
直观上来理解和接受,而不宜增加关于形变产生弹力的微观
解释。课本中说明弹力与重力不同,弹力只有在物体直接接
触并产生形变时才能产生。而实际上,有时微小形变又不易
察觉,这样从表观上就不易直接判别出相互接触的物体之间
究竞是否有形变与弹力产生。原则上这里只须提醒同学注意
这个问题,至于进一步具体判断的方法,应在以后学习物体的
平衡等节内容时再作讨论。有关弹力方向的问题,宜按课本
中的提法,即第13页和第14页两段有波纹线的文字,说明支
持力和拉力的方向,由于还存在扭转、切变等种种形变,教学
中不宜笼统地表述为“弹力的方向总是指向……”的形式。

\textbf{4.摩擦力}\quad 摩擦力一节的教学重点是滑动摩擦力,教学
时可在初中学过的知识基础上,通过演示得出关系式$f=\mu N$,
引出滑动摩擦系数的概念,对于这个公式,有的同学往往误
认为压力$N$的大小总是跟滑动物体所受的重力相等,教师
应该让他们知道,压力$N$是跟两个物体的接触面垂直的。只
有物体在水平拉力作用下沿水平面滑动时,压力$N$的大小才
跟物体所受的重力相等,在其他情况下,例如物体沿斜面下滑
时,压力$N$并不等于物体所受的重力。

摩擦力不是教学的重点,在判断静摩擦力的方向时,
同学们对相对运动趋势常常感到比较抽象。在不讨论静摩擦
力作为动力的情况下(如传送带上的物体等),可引导同学这样
来认识:按照已经给定的力来看,物体本是要运动的,但实际
上物体却处于静止状态,那阻碍物体运动的力便是静摩擦力。
这样静摩擦力的方向也就随之而明确了。

\textbf{5.作用力和反作用力}\quad 
牛顿第三定律是一个基本定律,
是本章的重点,讲好重点知识,应该引导学生抓住定律的主
要之点。对初学者来说,牛顿第三定律的主要之点就是作用
力和反作用力分别作用在相互作用的两个物体上,教材正是
抓住了这一点,通过实例和演示,反复加以说明,而没有侧重
于作用力和反作用力是同种性质的力,它们同时产生、同时消
失。在教学中应该注意这个问题,不能把主要之点当作自明
之理一带而过,把力量耗费在讲述一些次要问题上。讲解牛
顿第三定律,也是在进一步扩展学生对力的概念的认识,明确
力是物体对物体的相互作用。关于作用力和反作用力跟平衡
力之间的区别,学生常常理论上知道,实际上还会混淆。教学
中要通过一些实例,引导学生搞清两者的区别。例如,可以分
析放在桌面上的静止物体,找出它所受的一对平衡力,以及物
体所受的重力和它对桌面的压力的反作用力,还可以分析用
悬绳挂在天花板下的物体,找出作用在物体和悬线上的平衡
力,以及地球和物体,物体和悬绳、悬绳和天花板间的作用力
和反作用力。当然这个问题也不是一堂课所能解决的,下一
节物体受力分析,还要讨论这类问题。

\textbf{6.物体受力情况分析}\quad  这一节内容,是在以前各节预备
知识的基础上提出的,是前面各节有关知识的应用,这一节内
容的安排也是循序渐近的:从静止物体到运动物体,从平面上
的物体到斜面上的物体,从具体实例上升到分析物体受力情
况的一般方法——隔离法。

在教学上要注意:
\begin{enumerate}
    \item 引导学生正确地搞清楚研究对象,
施力物体与受力物体;
\item 要找到分析对象受到的所有的力,
不能遗漏,但也不能“无中生有”,不能“张冠李戴”;
\item 不能只
讲一般原则和注意事项,不能仅靠课堂上受力分析的示范,还
要行适量的实例练习,画出受力图,及时发现问题,及时引
导同学自觉纠正错误,逐步掌握正确的受力分析方法。
\end{enumerate}

物体受力情况是各种各样的,因此不可能在这一节教学
中要求学生完全掌握,要有一个过程。循序渐进在这里特别
重要,切不可一次就补充很多复杂的题目让同学分析。这样
反而会使同学无所适从,甚至产生畏难情绪。对隔离法的教
学要求尤宜如此,所以课本中没有提出连接体之类的繁难问
题,而把重点放在引导学生理解和掌握受力分析的方法和思
路上。如果学生能掌握正确的方法和思路,则他们自己也会
逐渐独立地解析各种力学题目。

在受力分析时,常常忽略某些次要因素,这也是使所研究
的问题理想化,应该在教学中引导同学重视并逐渐熟悉这种
方法。例如物体下落时,相对于重力来说,可以忽略空气阻力
这一次要因素,通过理想化才构成了自由落体运动这一模型。
以后的教学中还要不止一次地运用这种方法,使本来很复杂
的问题,能够较容易地入手研究。当然这些话并不是都要在
这节课中向学生一一说明的。

\subsection{第二单元}
这一单元讲力的合成和分解,主要使学生掌握力的平行
四边形法则。这个法则是进行力学计算的基本规律,是本章
的叉一重点,由于学生初次接触这种运算,很不习惯,因此也
是难点。教学中应该通过演示,实验和力的图示,使学生理解
和掌握这一规律。

\textbf{1.力的合成}\quad  
在讲合力和力的合成的概念时,首先要从
生活中的事例出发,让学生理解儿个力共同作用的效果可以
跟另外一个力单独作用的效果相同,例如,一件行李,可以由
两个人共同提,也可以由一个人提;吊起一个重物,可以用两
根悬绳,也可以只用一根悬绳;一辆车子,可以由几个人推着
它匀速前进,也可以由一个人推着它匀速前进,等等,然后再
引入合力和力的合成的概念,这可以使抽象的概念具体化,便
于学生理解.力的平行四边形法则的教学,做好课本中图1.
22的演示实验是个关键。可以把实验装置装在竖立的小黑
板上,边讲边画出力的图示,还要事先设计好几组不同的数
据,不要只由一次实验的结果就总结出规律来。在处理力的
三角形法时,可以只把它当作代替力的平行四边形法则的简
单的作图法,介绍三角形法时,要学生搞清楚“首尾相接”的
意思,以免把合力的方向搞错,有的同学对于合力跟分力之
间的关系认识不清,学了力的合成以后,往往认为物体在受
到几个力作用的同时,还要受到它们的合力的作用。这种错
误认识应该纠正,要使学生认识,合力与分力之间是等效“代
替”的关系,而不是合力跟分力同时作用在物体上。

\textbf{2.力的分解}\quad   在讲力的分解时,也要通过实例和演示,
使学生体会:一个力往往可以产生几个效果。例如,在同一悬
点上,用两根悬绳吊起一个物体,物体对悬点的竖直向下的
拉力,产生了同时拉紧两根悬绳的效果,由此可以引出分力
和力的分解的概念。

在讲述怎样分解一个力时,教材是通过两个实际的例子,
得出一个力可以根据它产生的效果进行分解的结论,在这里,
教材上的提法是谨慎的、留有余地的。因为在许多情况下,力
的分解不是根据它产生的实际效果,而是要按照研究问题的
方便来进行的,大家熟知的力的正交分解法就是一个很好的
例子。

\subsection{第三单元}
这个单元是在前两个单元的知识基础上,提出矢量的概
念,并学习同一直线上的矢量的运算方法,为以后直线运动中
位移和速度的合成提供了依据。

\textbf{1.矢量与标量}\quad  
这节教材的教学,主要在于讲清楚矢量
与标量的不同含义和不同运算法则。这里从力的合成要按照
平行四边形法则来进行,外推到平行四边形法则也是矢量合
成(矢量加法运算)的普遍法则就够了,无须再举例展开,矢
量与标量在运算规则上的不同,教学中也只举加法运算为例,
至于矢量的减法和乘、除运算就更不须提出来进行比较了。

\textbf{2.一维矢量运算}\quad   在同一直线上的矢量的运算,是矢量
运算中最简单的情况,在讲过一般的矢量加法运算之后,在这
里作较为详细讲述,是因为以后讲直线运动时要用到这一知
识,要着重讲清:先要沿着矢量所在的直线选定一个正方向,
规定凡是方向跟正方向相同的矢量都取正值,凡是方向跟正
方向相反的矢量都取负值。这样,就可以用一个带有正负号的
数值把矢量的大小和方向都表示出来,从而把同一直线上的
矢量运算简化为代数运算,为下一章直线运动中位移和速度
的运算提供了很大的方便。教学中应该引导学生重视这一节
内容并仔细阅读认真领会课文,要弄清楚课本上说的“可以用
一个带有正负号的数值把矢量的大小和方向都表示出来”的
含义,是把矢量的大小和方向分开来表示的.例如$F=-6$牛,
力的大小用数值和单位(6牛)表示,而力的方向则按跟预先
规定的正方向相同或相反(取正号或负号)来表示,掌提了
这个基本点,一维矢量的运算就容易弄清楚了。

\section{实验指导}
\subsection{演示实验}
\subsubsection{重心的实验测定方法}
在演示课本图1.4所介绍的用悬挂法测物体的重心位置
时,应使学生明确:
\begin{enumerate}
\item 利用这种方法测重心只能用于薄板状(即厚度极小
可以忽略)的物体,任何有厚度的物体的重心不会在物体的某
一个表面上。
\item 重心可以不在物体上,如用悬挂法可以测出一个薄
板状的塑料衣架的重心$C$并不在衣架上(图1.1)。一个薄板
状的圆环的重心一定在环心上。
\end{enumerate}

\begin{figure}[htp]
    \centering
\includegraphics[scale=.8]{fig/1-1.png}
    \caption{}
\end{figure}


\subsubsection{物体形变时产生的弹力}
在做课本图1.5的实验时,为了突出所观察的是一端固
定的弹簧被拉长和被压缩时所产生的弹力对小车的作用,小
车上不必加放砝码,以免分散学生的注意力。如果为了使效
果明显可用倔强系数较小的弹簧。

课本图1.6所示的现象,要用较大的、一侧透明的水槽,
圆木的质量又必须足够大,一般不容易演示,如果用投影仪,
则不易看清细木棍的弯曲形变的恢复过程,因此建议改用细
木条(用制作模型飞机的木条$1\x2\x300 {\rm mm}^3$)把一辆原来静
止的小车推开的现象来演示。

演示课本图1.7的现象时,可用洗澡用的或做沙发床垫
用的塑料,观察发生形变和形变的恢复,效果较好。

\subsubsection{显示微小形变的实验装置}
可按课本图1.10的装置进行演示,为了使效果明显,应
使两块平面镜$M$和$N$间的距离相隔得尽可能远些,屏幕可
利用教室的墙壁,以便离开平面镜$N$的距离更大些。如果用
激光作为光源,在教室内就可演示。如果用白炽灯制成的平
行光源,则需在作为屏幕的墙的一边,用黑窗帘(或黑纸)把窗
户局部遮光。

这一装置是利用光在均匀媒质中的直线传播和平面镜组
对于光线的二次反射原理制成的,在入射光线的方向不改变
的情况下,如果平面镜转过$\theta$角,则反射光线将偏转$2\theta$角(图
1.2),当在桌面上施加一个压力,桌面发生微小的弯曲形变
时,两块原来平行放置的平面镜就不再平行了,$M$将向左侧















