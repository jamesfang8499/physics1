
\chapter{物体的平衡}\minitoc[n]
\section{教学要求}
这一章讲述静力学的基本知识,主要是讲在共点力作用
下物体的平衡和有固定转动轴的物体的平衡。

这一章的教学要求是:
\begin{enumerate}
\item 理解平衡和平衡条件的概念,掌握在共点力作用下物
体的平衡条件。
\item 理解力矩的概念,掌握有固定转动轴物体的平衡
条件。
\item 了解力偶和力偶矩的概念,知道力偶的作用是使物体
只发生转动。
\item 了解平衡的种类和稳度。
\end{enumerate}

下面对这一章的教学内容作些具体说明。

在动力学之后讲述静力学,有可能把静力学知识当成动
力学的特殊情况来理解,课本在实验的基础上得到共点力作
用下物体的平衡条件后,再从牛顿第二定律推导出这个平衡
分件,其目的就是要加深学生对平衡条件的理解。

讲述力对物体的转动作用,是为了讲解有固定转动轴的
物体的平衡做准备,这里讲述的转动的特点、转动快慢的描
述以及匀速转动和变速转动的概念,要求学生了解即可。关
于力矩,重点是要求学生理解力矩的概念,至于力矩的代数
和等于零或不等于零时物体怎样转动,使学生有个了解就可
以了。

关于有固定转动轴的物体的平衡条件的教学,还是以实
验为基础,但应注意实验与推理相配合,以加深学生的认识。

力偶是一个重要概念,而且以后讲通电线圈在磁场中运
动时会用到,因此应使学生对力偶知识有所了解。这里主要
是使学生明确知道力偶与一个力对物体的作用不同,--个力
可以使物体同时发生转动和平动,而力偶的作用是使物体只
发生转动而不发生平动。对有固定转动轴的物体来说,虽然
一个力的作用和力偶的作用都可以使物体发生转动,但效果
是有区别的,教材中说明这一点,是为了加强力偶的作用是
使物体只发生转动的认识。

关于平衡的种类和稳度的教学,都只要求作一般的介绍,
目的是扩展学生的知识面,把他们的静力学知识用来分析这
一常见现象。平衡的种类的区分,要让学生知道区分的标志
是以物体的平衡遭到破坏之后能否自行回到原来的平衡
位置。讲解稳度,要使学生知道在必要时增大稳度的方法。

理论联系实际,是物理教学的一条重要原则,静力学知
识在实际中很有意义,这一章的教学更应注意与生产和生活
实际的联系,所选习题力求有实际意义,而不要补充那些过难
而又缺少实际意义的题。

\section{教学建议}
本章教学建议分成四个单元,第一单元(全章引言和第
一节)主要讲述什么是物体的平衡状态和平衡条件,并得出
在共点力作用下物体的平衡条件。第二单元(第二、三节)
通过力对物体的转动作用的讨论,引入力矩这个重要概念,并
进一步得出具有固定转动轴的物体的平衡条件。第三单元
(第四节)讲述力偶的初步知识。第四单元(第五、六节)是对
物体平衡状态的进一步分析,说明物体平衡还有个稳定不稳
定的问题,稳定平衡还有个稳定程度的问题。

\subsection{第一单元}
本单元教学应该以实验为基础,做好三个共点力平衡的
演示实验。但同时要注意实验与推理相结合,这里主要是两
方面的推理:一是将三个共点力的平衡条件推广到三个以上
共点力作用下的平衡条件。虽然教材说“用实验还可以证明”,
但教学时一般不必再做三个以上共点力平衡的实验了。另一
个是从牛顿第二定律推出共点力作用下物体的平衡条件。

\subsubsection{对平衡状态的理解}

在全章引言中,教材明确指出:
“如果一个物体既不做平动,也不做转动,即保持静止,或者
做匀速直线运动或匀速转动,我们就说这个物体处于平衡状
态。”这是平衡状态的定义。比起以前教材中关于物体平衡的
说法,含义更广了。以前所说的平衡,仅指物体处于静止或匀
速直线运动状态。而这里的定义说,匀速转动状态也是平衡
状态。为什么说匀速转动也是平衡状态呢?因为它与静止和
匀速直线运动有共同之处,即运动状态保持不变。前者是物
体的即时速度保持不变(加速度为零),后者是物体转动的角
速度保持不变,这样,既使学生对平衡状态概念的理解进一步
深化,而且为利用牛顿第二定律推导出共点力平衡条件,也为
以后推导有固定转动轴的物体的平衡条件,作了准备,教材
中把物体的平衡安排在牛顿运动定律之后讲述,也为加深对
平衡这一概念的理解提供了条件。

\subsubsection{什么是共点力}

在教学中可以提一提,所谓共点力并
不一定是几个力都作用在同一点上,还应包括几个力的作用
线相交于一点的情况。因为作用在物体上的力沿着力的作用
线平移,其作用效果是相同的。此外还应注意,我们在这里所
讨论的共点力,仅限于在同一平面上的共点力,有的书上称为
共面力。如果不是共面力,情况就要复杂得多。

\subsubsection{三力平衡实验}

证明三个共点力作用下的平衡条件
是合力为零的实验并不难做,但要得出平衡条件则必须把实
验做得尽量准确一点.建议在教学中注意两点:
\begin{enumerate}
\item 课本图
6.1甲中的整个装置应是水平放置的.如果竖直放置,则可
能因下面一个弹簧倒置而产生较大误差。    
\item 如果教师演示
时,不便将装置水平放置,则可以将下面那个弹簧秤牵拉的力
改为挂上砝码,当然,受力物体应该选择重量很小的物体,或
者就可以把绳子的结点作为受力平衡的研究对象。
\end{enumerate}

\subsubsection{平衡条件的应用}

共点力的平衡条件$F_{\text{合}}=0$原是一
个矢量方程,由于教材不介绍正交分解法,所以必须设法将
平衡条件简化,例如本章后习题1的解答中,先将$F_1$和$F_2$
合成为$F'$, 接着就把方程$F_{\text{合}}=0$简化为$F'$与$F$大小相等、方
向相反的结论,再通过几何关系分别求出$F_1$和$F_2$. 如何将
平衡条件$F_{\text{合}}=0$简化成既符合题意,又便于研究的形式,是学
生学习静力学时常常感到困难的问题,教学中要有意识加以
指导。

\subsubsection{静力学问题宜用平衡条件解}

学生在第一章学习力
的分解时,实际也解过类似本章后习题1的题目.所以学生
可能习惯于用力的分解来解题。在比较简单的情况下,从力
的作用效果出发,用力的分解来解题也是可以的,例如本章后
第1题,但如果题目复杂一些,用力的分解来解则可能发生
差错。因为有时某个力在某方向上的分力是没有实际意义的。
所以学了这一单元以后,建议学生以后可尽量用平衡条件来
解静力学问题。


\subsection{第二单元}
本单元讲述力对物体的转动作用、力矩的概念和有固定
转动轴物体的平衡条件。
\subsubsection{物体转动与物体平动的对应关系}

物体转动与质点
运动虽然是两种不同的运动,但它们的许多概念和规律都有
类似的关系,可在适当的时候进行对照比较。例如课本中所
说的,平动物体上各点的速度都相同,任何一点的速度可以
代表整个物体的速度。同样,转动物体上各点的角速度都相
同,任何一点的角速度都可以代表整个物体的角速度。这样,
平动物体的速度与转动物体的角速度相对应,另外还有,平动
的位移和转动时转过的角度(称为角位移)相对应,匀速直线
运动与匀速转动相对应,共点力的平衡条件与力矩平衡条件
相对应,等等。教学中教师可以启发学生进行这样的对比,有
利于学生对转动知识的理解。

\subsubsection{力矩的概念和计算}
学习力矩的概念,关键在于掌握
力臂的概念。关于力臂的意义和计算,虽然在初中学习杠杆
平衡条件时已经反复练习过,但在高中阶段学生还是常常容
易搞错。所以这里仍要多举几个例子让学生学会确定力臂的
长度。在此基础上,高中还要学生掌握力矩的单位和正负。
力矩的单位是${\rm N\cdot m}$,不是${\rm J}$,要求学生不要同功的单位混
淆。力矩有正负,但在教学中不要提力矩的方向。因为力矩
方向的规定涉及矢量乘法,已超出高中物理范围。

\subsubsection{关于有固定转动轴物体平衡的实验和推理}

通过实验
和推理得出有固定转动轴物体的平衡条件是本单元的重点。
课本是先通过推理得出平衡条件,然后用实验来验证的。教
学中如果教师采用先实验再说理的办法也是可以的。经过实
验,学生对力臂的计算和有固定转动轴物体的平衡条件将有
较深刻的理解.在将结论$M_1+M_2=M_3$改写成$M_{\text{合}}=0$的过
程中,教师还得作一些解释。前者等号两侧都是绝对值,等号
左侧为使物体向顺时针方向转动的力矩,右侧为使物体向逆
时针方向转动的力矩。后者$M_{\text{合}}$中包括了正负两种力矩。写
成$M_{\text{合}}=0$的形式便于同共点力平衡条件$F_{\text{合}}=0$对照.$F_{\text{合}}=0$
为一矢量方程,而$M_{\text{合}}=0$中的力矩不是正,就是负,是个代数
方程.因此在运用上$M_{\text{合}}=0$要比$F_{\text{合}}=0$容易掌握一些。

\subsubsection{关于固定转动轴的理解}

课本在对固定转动轴的理
解上作了一些扩展,即不限于研究确有实际的固定转轴的情
况,而把在研究问题时该物体绕某线转动的那条线看作是固
定转动轴,本章末习题第4题和第9题就是这样,解这类
题可以培养学生灵活应用所学知识的能力,但课本对一般平
面力作用下物体平衡问题,即须同时使用平动平衡条件和转
动平衡条件的问题一概不作要求,教师应加以注意。

\subsubsection{三角支架问题在什么情况下要用力矩平衡来解}

在支架本身重量可以忽略不计的情况下,三角支架问题可以用
共点力平衡条件来解,也可以用具有固定转动轴的物体的平
衡条件来解,但当支架本身重量不能忽略时,如本章后习题
第3题,则必须以横梁$BO$为研究对象,以$B$为转动轴,利用
力矩平衡条件来解.如果此时我们仍以$O$点为研究对象,$O$
点除受钢绳$AO$的拉力$T$和重物的拉力$F$以外,一定还有一
个$BO$对$O$点的支持力$N$. $T$、$F$、$N$三个力是共点的,$O$点又处
于静止状态,因此这三个力也一定满足$F_{\text{合}}=0$这个条件,但
是,当$BO$梁的自重不能忽略时,它对$O$点的支撑力$N$不再
沿着$BO$的方向.$N$的方向未定,用$F_{\text{合}}=0$来解,条件就不充
足了,如果我们用有固定转动轴的平衡条件先求出$T$的大小,
然后再利用共点力平衡条件可以求得$BO$对$O$点的支撑力
$N$, 很容易看出这个力不在$BO$的方向上。对程度较好的学
生,启发他们作以上的分析,对培养学生分析问题的能力很有
好处。


\subsection{第三单元}
这一单元介绍力偶这个概念,教学中应突出两个主要问
题:第一是使学生掌握力偶、力偶臂、力偶矩的意义,力偶矩大
小的计算和正负的规定;第二是要使学生明确,力偶的作用
是使物体只发生转动而不发生平动。
\subsubsection{关于力偶的作用}

力偶的作用是使物体只发生转动
而不发生平动,这个结论是通过力偶对物体的作用和一个力
对物体的作用的对比实验来使学生认识的.课本中图6.11和
图6.12的实验可利用书写幻灯进行投影增大可见度,被拉
动的物体可用一块圆形的有机玻璃板,上面画一些辐条,这
样在书写幻灯的投影下,可以清楚地看到平动和转动。课本
第三段说到用一只手板套筒容易磨损螺纹,用两只手板套筒,
轴上就不会受到压力,目的也是要突出力偶只使物体发生转
动.这一点,也可以利用课本图6.11和图6.12的装置加以
说明。具体做法见本章实验指导。

\subsubsection{力偶矩也是力矩}

教学中要使学生明白,力偶矩实际
上就是力偶的两个力的力矩之和。力偶矩和力矩的作用都是
使物体的转动状态发生变化,因此,不要把力偶矩与力矩这
两个概念孤立开来。教学中,要通过力矩代数和的计算推导出
力偶矩$M=Fd$的关系式。通过推导,还要启发学生理解力
偶矩的大小不随所取的转动轴位置的改变而改变。至于在力
偶的作用下物体究竟以哪一点为轴转动的问题,课本没有提,
教学中也应回避。


\subsection{第四单元}
本单元对物体平衡的问题作进一步的分析,目的是扩展
学生的知识面,把物理知识与实际联系得更紧密。教师在教
学过程中,应多举些实例,多做些演示,以加深对教材的理解,
培养学生观察和思考问题的能力。

本单元的重点是要学会怎样分析稳定平衡、不稳平衡和
随遇平衡,而不是光记忆这些结论,对有支点的物体来说,是
根据它偏离平衡位置后,重力和支持力的合力是否能
使物体回到平衡位置来判断的;对有支轴的物体来说,是
根据它偏离平衡位置后,合力矩能否使物体回到平衡位置来
判断的。或者两者都用重心的升高或降低来判断。教师在分
析各种实例时,都要从这个角度来分析。

在所举的实例和演示中,既要注意简単叨了,便于分析,
联系实际,也要注意生动活泼,活跃课堂气氛,例如不倒翁的
例子就比较生动。此外,杂技团演员走钢丝、顶碗等实例都有
个平衡和稳定问题。有条件的学校,可以放映一些科技电影,
如“杂技团的秘密”等。

\section{实验指导}
\subsection{演示实验}
\subsubsection{共点力作用下物体的平衡条件}
课本图6.1所示的演示实验是在水平面上做的,可
以通过投影仪显示,物体可用
有机玻璃做,这样,可以看到作
用点。


还可以利用两个弹簧
秤和一组钩码来进行演示,如
图6.1所示,要使学生认识这
个实验中的研究对象是绳的结
点$O$(也可以在$O$点固定一块用泡沫塑料锯成任意形状的薄
板作为象征性的物体),$O$点受三个共点力的作用而保持平
衡。如果调节好两个弹簧秤的位置,使得$F_1=120$克力.
$F_2=90$克力,$F_1$和$F_2$的夹角为$90^{\circ}$, 所用的钩码的总重量为150
克力。就可以方便地得出在共点力作用下,物体的平衡条件
是合力等于零。

\begin{figure}[htp]\centering
    \begin{minipage}[t]{0.48\textwidth}
    \centering
    \includegraphics[scale=.7]{fig/6-1.png}
    \caption{}
    \end{minipage}
    \begin{minipage}[t]{0.48\textwidth}
    \centering
    \includegraphics[scale=.7]{fig/6-2.png}
    \caption{}
    \end{minipage}
    \end{figure}

如图6.2所示,一辆小车放在光滑斜面上,小车一端
通过细绳和弹簧秤相连,弹簧秤的另一端固定在斜面顶端,当
小车静止时,小车受到三个力而平衡,使斜面倾角改变,可以
看到弹簧秤的示数也发生相应的变化,达到某一数值时,小车
又处于平衡状态。



\subsubsection{力矩的作用}

\begin{figure}[htp]
    \centering
    \includegraphics[scale=.7]{fig/6-3.png}
    \caption{}
\end{figure}

可以利用教室的门做如图6.3所示的演示,把橡皮绳的
一端系在门把手上,另一端捏在手中,先将门关上,沿着垂
直于门面的方向拉动橡皮绳,观察到橡皮绳稍有伸长,门就
能被拉开(如图中橡皮绳的拉力为$F_1$);然后重新把门关上,
改变拉像皮绳的方向,使得拉力仍沿水平方向但不与门面
垂直。这时可观察到橡皮绳必须拉得更长,才能使门拉开(如
图中橡皮绳的拉力增大为$F_2$)。启发学生思考,为什么会出
现这种现象?说明什么问题?然后演示橡皮绳在水平方向沿着
门面拉动,直到橡皮绳被拉断,还是拉不开门,让学生思考这
又是为什么?

\subsubsection{有固定转动轴的物体的平衡条件}
正力矩和负力矩的规定:
根据课本图6.8, 用力矩盘进行演示,指出力$F_1$和$F_2$所产
生的力矩是负的,它们的作用效果是使有固定转动轴的物体
向顺时针方向转动、力$F_3$产生的力矩是正的,它的作用效果
是使物体向反时针方向转动。

仍利用课本图6.8所示的力矩盘进行演示,并可改
变细绳下端所挂钩码的个数,改变作用点的位置。当力矩盘
平衡时,从各个力和力臂的大小可以得出:有固定转动轴的物
体的平衡条件是力矩的代数和等于零。

\subsubsection{应用有固定转动轴的物体的平衡条件解题}
课本213页习题第4题,可用模拟演示来加以说
明,如图6.4所示,将一根粗细
不均匀的木棍(如教棒),平放
在讲台上,在棍的两端各系一个细绳套,用弹簧秤先后勾住
棍的细端和粗端的绳套,稍稍提起,使该端脱离桌面,分别读
得弹簧秤的示数为$F_1$和$F_2$. 再用弹簧秤勾住绳套,把木棍整个
提起来,读出弹簧秤示数为$F$。从实验结果可知:$F=F_1+F_2$
\begin{figure}[htp]\centering
    \begin{minipage}[t]{0.48\textwidth}
    \centering
    \includegraphics[scale=.7]{fig/6-4.png}
    \caption{}
    \end{minipage}
    \begin{minipage}[t]{0.48\textwidth}
    \centering
    \includegraphics[scale=.7]{fig/6-5.png}
    \caption{}
    \end{minipage}
    \end{figure}

课本213页第9题可用铁架台及顶部装有定滑轮
的木杆制成模型进行演示(图6.5)。


当在定滑轮下所挂的重物
$G$的重量过大时,铁架台将向
右倾倒,可以减小重物$G$的重
量或缩短拉紧撑杆的细绳长
度,使得撑杆的倾角变大,直到
铁架台恰巧不倾倒。应该指出,
在这个模拟演示中是将整个铁
架台连同撑杆和所挂重物看成
一个物体来进行研究的,课本213页第9题的起重机也应
该看作是一个整体,各个力对前轮$O$所产生的力矩代数和应
等于零,来求出起重机至多能提起多重的物体。

\begin{figure}[htp]
    \centering
    \includegraphics[scale=.7]{fig/6-6.png}
    \caption{}
\end{figure}

如图6.6所示,将均
匀米尺的一部分伸出水平桌面
外,在米尺伸出部分的顶端放
一小砝码(10克或20克),调节
米尺伸出部分的长度,使得米尺仅对桌边有压力,这时就可以
把米尺跟桌边接触的地方看成是固定转动轴。只要读出伸出
部分的米尺长度,计算从均米尺的重心位置到桌边的距离,
从已知砝码的重量,根据有固定转动轴物体的平衡条件,即可
求出米尺的重量。

\subsubsection{力偶}
课本图6.11和图6.12的演示可以通过投影仪来显
示,圆盘可以用厚一些的有机玻璃板来做,在圆盘的侧边车制
一凹槽,以便绕线。为了使效果明显,在有机玻璃圆盘上沿着
半径方向可用透明漆画几条有色条纹。

\begin{figure}[htp]
    \centering
    \includegraphics[scale=.7]{fig/6-7.png}
    \caption{}
\end{figure}

如图6.7所示,在铁架台上通过复夹和试管夹安装
一块水平放置的中央有孔的有
机玻璃板,在平板上放一个准
备好的有机玻璃圆盘。在盘心
位置开一个圆孔,将一细竹针
穿过圆孔,并用复夹使竹针的
两端固定。然后在圆盘的一侧
通过缠绕的细线拉圆盘,可以
观察到圆盘转动的同时,竹针弯曲,这说明力矩的作用可以使有固定转动轴的物体发生转
动,但转动轴是受力作用的。如果不存在固定转动轴(将竹针
抽去),在力矩的作用下圆盘将同时发生转动和平动。如果对
圆盘作用一个力偶则可观察到圆盘转动时竹针并不弯曲,说
明竹针并不受力。因此即使抽去竹针,在力偶作用下,圆盘也
只发生转动,不发生平动。

\subsubsection{物体平衡的种类}
课本图6.16有支点的物体的平衡的演示,可利用小球放
在离心轨道(间距小于小球直径的两根平行铁丝)上来进行演
示.课本图6.17有支轴的物体的平衡,可用一均匀薄木板
(厚度约为2—3mm)来演示,中间的小孔要开在薄木板的重
心上,孔内侧要粗糙些,这样,演示随遇平衡的效果会好些。


\subsubsection{稳度跟重心的高低和支面的大小有关}
可制成一个如图6.8所示的形状可变的框架(稳度演示
器)来进行演示,当把它由长
方体改变成斜方体时,只要系
在它的中心(表示重心位置)的
重垂线不超出支面,则斜方体
就不会倾倒。
如果重垂线超出
支面,斜方体就会倾倒。

\begin{figure}[htp]
    \centering
    \includegraphics[scale=.7]{fig/6-8.png}
    \caption{}
\end{figure}

\subsection{学生实验}
\subsubsection{研究有固定转动轴物体的平衡条件}

实验时,先用胶纸在力矩盘上粘贴一张白纸,用手指
隔着白纸在转动轴部位按一下,在纸上留下转动轴的痕迹,然
后按课本图10.15所示的装置把力矩盘装好.要注意转动轴
应在承平方向,使力矩盘位于竖直平面内。在盘上任意选择四
个位置,各插一根大头针(要插深些),再按课本的要求(课本图10.15)在三根针上用细线悬挂钩码,悬挂细线时要造当
靠近大头针的根部,但又不要使细线和盘面发生摩擦。在第
四根针上用细线钩在弹簧秤的钩上,要注意调节固定在横杆
上的弹簧秤的位置,使得力矩盘平衡时,弹簧秤的拉线不要通
过转动轴。

当力矩盘在这四个力作用下处于平衡状态时,记下
这四个力的大小。用削细的铅笔沿着四根细线的方向,在离开
大头针较远的地方,分别画上一个“$\x$”号,并在悬挂这些细线
的大头针的针孔周围画一小圆,以便确定这四根拉力的方向。

取下钩码和弹簧秤,拔去大头针,取下白纸;用直尺
将做过记号的针孔和相关的“$\x$”号用虚线连接起来,并根据
自己选定的标度(要在记录纸上明确标出),按力的图示法分
别画出这四个拉力$F_1$、$F_2$、$F_3$和$F_4$.

根据事先在白纸上所做的记号,画出固定转动轴$O$,
然后分别作出从$O$点到四个拉力作用线的垂线,并用毫米刻
度尺量出这些垂线的长度,这就是力臂,在原始记录纸上标出
各个力臂$L_1$、$L_2$、$L_3$和$L_4$.

这个实验中也可以不
用横杆来固定弹簧秤,而用另
一个铁架台,通过复夹用试管
夹把弹簧秤背面的铁壳夹紧
(图6.9),这样做的好处在于可
以方便地调节弹簧拉长的方
向,使它跟细线的方向一致。

\begin{figure}[htp]
    \centering
    \includegraphics[scale=.7]{fig/6-9.png}
    \caption{}
\end{figure}

启发学生讨论课本最后提出的问题,认识到这是为
了便于使力矩盘平衡。因为弹簧秤在它的量程范围内,拉力的。
大小可以连续变化,这样就可以自行改变弹簧秤拉力的大小,
使得力矩盘平衡。

\subsection{课外实验活动}
\subsubsection{制作杆秤}
在练习制作时,可注意并思考以下几点:
\begin{enumerate}
\item 制作杆秤所用的细木棍要挑比较结实一些的材料,但
不要太粗,如果找不到合适的材料,找一根较长的竹筷或木筷
也可以。提纽要用强度大一些的细绳,秤钩可以用粗铁丝制
成,(也可以用几股细铁丝绞起来做),秤锤用质量小于1千克
的物体也可以。
\item 在秤钩不挂物体的情况下,把秤锤挂在秤杆上确定
秤杆的零刻度(课本图6.26的$A$点)时,要注意秤杆
和秤钩(看成一体)的重心在提纽$O$点的哪一侧?
上
\item 证明秤杆上刻度间的距离为什么是均匀的。可以结
合课本214页第10题的计算进行讨论.
\item 要增大杆秤的称量范围,可以再装一个提纽(称做二
纽),想想看这个二纽的位置应该离秤钩远一些还是近一些?
\end{enumerate}


\section{习题解答}

\subsection{练习一}


\begin{enumerate}
    \item 在课本第一章图1.21所示的情形里,如果物体的重量是40牛,绳子$a$与竖直方向成30$^\circ$角,绳子$a$和$b$对物体的拉力分别是多大?

    \begin{solution}
        设物体重量为$G$, 绳子$a$、$b$对物体的拉力分别是$F_1$、
        $F_2$. 作物体受力图如图6.10所示.这是一个共点力平衡的
        问题.根据平衡条件可知,$F_1$、$F_2$的合力$F$一定与$G$大小相
        等、方向相反,在$\triangle OFF_1$中,
        $\dfrac{F}{F_1}=\cos30^{\circ}$, 而$F=G$, 所以
\[F_1=\frac{G}{\cos 30^{\circ}}=\frac{40}{\sqrt{3}/2}=46.2{\rm N}\]
同理:
\[F_2=F\tan 30^{\circ}=G\tan 30^{\circ}=40\x \frac{\sqrt{3}}{3}=23.1{\rm N}\]    
    \end{solution}

    \begin{figure}[htp]\centering
        \begin{minipage}[t]{0.48\textwidth}
        \centering
    \includegraphics[scale=.7]{fig/6-10.png}
        \caption{}
        \end{minipage}
        \begin{minipage}[t]{0.48\textwidth}
        \centering
    \includegraphics[scale=.7]{fig/6-11.png}
        \caption{}
        \end{minipage}
        \end{figure}

    \item 把物体放在光滑的斜面上,并用弹簧把它拉住,如图6.11所示.如果物体的质量为$m$,斜面的倾角为$\theta$,弹簧对物体的拉力是多大?


    \begin{solution}
        物体受三个力:重力$G=mg$、斜面支持力$N$、弹簧
        拉力$f$. 根据平衡条件,$N$与$f$的合力$F$应与$G$的大小相同、
        方向相反。所以
     \[   f=F\sin\theta= mg\sin \theta\]
    \end{solution}


    \item 图6.12是起重机匀速起吊重物时吊钩的受力情况.吊钩受到竖直向上的牵引力$F_1$和两条钢索对它斜向下的拉カ$F_2$和$F_3$.如果两条钢索的夹角为60$^\circ$,力$F_2$和$F_3$大小相等,力$F_1$为$2.0\times 10^4$牛,每钢索对吊钩的拉力是多大?

\begin{figure}[htp]\centering
\begin{minipage}[t]{0.48\textwidth}
\centering\includegraphics[scale=.7]{fig/6-12.png}
\caption{}
\end{minipage}
\begin{minipage}[t]{0.48\textwidth}
\centering\includegraphics[scale=.7]{fig/6-13.png}
\caption{}
\end{minipage}
\end{figure}

\begin{solution}
已知$F_1=2.0\x10^4$牛,$F_2=F_3$, 夹角$\angle AOB=60^{\circ}$. 因
为是匀速起吊,所以吊钩$O$受力平衡.$F_2$和$F_3$的合力$F$与
$F_1$大小相等方向相反.从图6.12中可知,四边形$OF_3FF_2$是
个菱形.作菱形的对角线$F_2F_3$, 与$OF$交于$O'$点,在三角形
$OO'F_2$内:
\[F_2=F_3=\frac{F_1/2}{\cos30^{\circ}}=\frac{2.0\x 10^4}{\sqrt{3}}=1.2\x 10^4{\rm N}\]
\end{solution}

    \item 掘沟机由两台拖拉机牵引(图6.13),两条绳索对掘沟机的拉力都是$2.5\times 10^4$牛,绳索间的夹角为60$^\circ$,如果拖拉机是匀速行进的,土地的阻力有多大?

    \begin{solution}
把掘沟机作为研究对象,它的受力图如图6.13所
示,$f$为土地的阻力。因为掘沟机匀速行进,受力平衡,根
据平衡条件,两绳拉力的合力$F_{\text{合}}$与$f$大小相等、方向相反。
应用余弦定理,有
\[F^2_{\text{合}}=F^2+F^2+2F^2\cos60^{\circ}=2F^2(1+\cos60^{\circ})\]
所以
\[f=F_{\text{合}}=F\sqrt{2(1+\cos60^{\circ})}=\sqrt{3}F=\sqrt{3}
\x2.5\x10^4=4.3\x10^4{\rm N}\]
    \end{solution}


    \item 如图6.14所示,物体在五个共点力的作用下保持平衡,如果撤去力$F_5$,而保持其余四个力不变,这四个力的合力的大小和方向是怎样的?
\begin{figure}[htp]
\centering \includegraphics[scale=1.3]{fig/6-5.PDF}
\caption{}
\end{figure}

\begin{solution}
    根据共点力的平衡条件,$f_1$、$f_2$、$f_3$、$f_4$的合力$F$必定
    与$f_5$大小相等、方向相反。
\end{solution}
\end{enumerate}



\subsection{练习二}

\begin{enumerate}
    \item 当我们开关门窗时,如果力的作用线通过转轴,无论多大的力也不能把门窗打开或关上,为什么?

  \begin{solution}
        力的作用线通过转轴时,力臂为零,力矩就为零,这
        个力对门窗不产生转动效果,所以不能把门窗打开或关上。
    \end{solution}
  
    \item 如图6.15所示,如在自行车脚踏板上的向下的力是15牛,求这个力的力矩.
    \begin{figure}[htp]\centering
        \begin{minipage}[t]{0.48\textwidth}
        \centering
    \includegraphics[scale=.7]{fig/6-15.png}
        \caption{}
        \end{minipage}
        \begin{minipage}[t]{0.48\textwidth}
        \centering
    \includegraphics[scale=.7]{fig/6-16.png}
        \caption{}
        \end{minipage}
        \end{figure}


\begin{solution}
    设$\ell$为力$F$对轴$O$的力臂,则$\ell =17.5\x\cos30^{\circ}=15.2{\rm cm}=0.152{\rm m}$,力矩$M=F\ell=15\x0.152=2.28{\rm N\cdot m}$
\end{solution}

    \item 图6.16是汽车制动器的路板的示意图.$O$是转动轴,$B$端连接制动器.如果司机踏紧踏板的力$F$为20牛,制动器的阻力$F'$是多大?

\begin{solution}
    踏板是一个具有固定转动轴的物体,设$F$的力臂为$\ell$, $F'$的力臂为$\ell'$, 则根据平衡条件,有
    $F\ell=F'\ell'$, 则
\[F'=\frac{F\ell }{\ell'}=\frac{20\x 0.21}{0.12}=35{\rm N}\]
此题的力矩平衡条件也可写成
\[M+M'=0\]
式中$M=F\ell$, 使物体作逆时针方向的转动,为正值;$M'=
F'\ell'$, 使物体作顺时针方向的转动,为负值。若$M$、$M'$均以绝
对值代入,则$M-M'=0$. 即
\[20\x0.21-F'\x0.12=0\]
解得$F'=35$牛.
\end{solution}

    \item 试根据有固定转动轴的物体的平衡条件来证明初中学过的杠杆的平衡条件:如果作用在杠杆上的两个力使杠杆向相反方向转动,并且这两个力的大小跟它们的力臂长度成反比,杠杆就平衡.
    \begin{figure}[htp]
        \centering   
\begin{tikzpicture}[>=latex]
    \draw(-2,-.1)node[above=3pt]{$A$} rectangle (4,.1)node[above]{$B$};
\draw(0,-.1)--(-.1,-.3)--(.1,-.3)--(0,-.1);
\node at (0,.1)[above]{$O$};
\draw[->, thick](-2,-.1)--(-2,-1.5)node[right]{$F_1$};
\draw[->, thick](4,-.1)--(4,-1)node[right]{$F_2$};

\end{tikzpicture}
        \caption{}
        \end{figure}

    \begin{solution}
        设有一杠杆如图6.17所示,根据有固定转动轴
        物体的平衡条件,应该有$M_1+M_2=0$.

        考虑到$M_1=F_1\x OA$, 为正值.$M_2=F_2\x OB$, 为负值。
        所以式中$F_1$、$F_2$、$OA$、$OB$均取绝对值,则得$F_1\x OA-F_2\x
        OB=0$. 变形后得
        \[\frac{F_1}{F_2}=\frac{OB}{OA}\]
        这就是杠杆平衡条件。
    \end{solution}
    \item 一个有固定转动轴的物体受到四个力的作用,其中使物体向顺时针方向转动的两个力是5牛和3牛,使物体向反时针方向转动的两个力是2牛和6牛.这四个力的力臂依次是0.50米、0.25米、0.05米、0.20米,在这四个力的作用下,物体能否平衡?为了使物体平衡,应该把一个1牛的力加在物体上离转动轴多远的地方?这个力的力矩是正的还是负的?

    \begin{solution}
根据题意,已知$F_1=5$牛、$F_2=3$牛、$F_3=2$牛、$F_4=
6$牛、$\ell_1=0.50$米、$\ell_2=0.25$米、$\ell_3=0.05$米、$\ell_4=0.20$米.

使物体顺时针转动的力矩
\[M_{\text{顺}}=M_1+M_2=-5\x0.50-3\x0.25=-3.25{\rm N\cdot m}\]
使物体反时针转动的力矩
\[M_{\text{反}}=M_3+M_4=2\x0.05+6\x0.20=13{\rm N\cdot m}\]

$M_{\text{顺}}+M_{\text{反}}\ne 0$, 所以不能平衡.再加一个1牛的力$F_5$, 使$M_5+M_{\text{顺}}+M_{\text{反}}=0$, 则
\[M_5-3.25+1.3=0\quad \Rightarrow\quad M_5=1.95{\rm N\cdot m}\]
\[\ell_s=\frac{M_s}{F_s}=\frac{1.95}{1}=1.95{\rm m}\]
1牛的力应加在离轴1.95米的地方.这个力矩是正的.
    \end{solution}
\end{enumerate}



\subsection{练习三}
\begin{enumerate}
    \item 在图6.18中,汽车方向盘的半径是0.20米,司机两手加在方向盘上的力都是15牛,求方向盘受到的力偶矩.
    \begin{figure}[htp]
        \centering    \includegraphics[scale=.6]{fig/6-18.png}
        \caption{}
        \end{figure}

    \begin{solution}
        已知$R=0.20$米,$F=15$牛,则力偶矩
\[M=F\x2R=15\x 0.40=6.0{\rm N\cdot m}  \]
    \end{solution}
    \item 一个物体受到两个力偶的作用,力偶矩的代数和50牛·米,这两个力偶的四个力的力矩的代数和是多大?

   \begin{solution}
        也是50牛·米.因为力偶矩的大小实际上就是两个
        力的力矩的代数和,而且它的太小同转动轴的位置无关。所
        以两个力偶矩的代数和为50牛·米,它们四个力的力矩的代
        数和也一定是50牛·米. 
    \end{solution}
    \item 一个物体受到三个力偶的作用,其中两个使物体向顺时针转动的力偶的力分别是3牛和4牛,力偶臂分别是0.50米和0.25米;另一个使物体向反时针转动的力偶的力是5牛,力偶臂是0.50米,这个物体能够平衡吗?


    \begin{solution}
        设三个力偶的力分别为$F_1$、$F_2$、$F_3$, 力偶臂分别为
        $d_1$、$d_2$、$d_3$. 根据题意,$F_1=3$牛、$F_2=4$牛、$F_3=5$牛、$d_1=0.50$
        米、$d_2=0.25$米、$d_3=0.50$米,它们的力偶矩为
        \[\begin{split}
            M_1&=-F_1d_1=-3\x0.50=-15{\rm N\cdot m} \\
        M_2&=-F_2d_2=-4\x0.25=-1.0{\rm N\cdot m} \\
        M_3&=F_3d_3=5\x0.50=2.5{\rm N\cdot m}       
        \end{split}\]
        $M_1+M_2+M_3=0$. 可见,这个物体能够平衡.
    \end{solution}
\end{enumerate}




\subsection{练习四}
\begin{enumerate}
    \item 背上背着重东西的人,为什么要向前倾?


    \begin{solution}
        如果身体不向前倾,则背上重物后人的重心要后移,
        有可能使重刀作用线超越支面,使人向后跌倒。人体向前倾,
        可使重力作用线仍在支面内,保持人体平衡。
    \end{solution}
    \item 老年人扶手杖走路为什么不容易跌倒?


    \begin{solution}
        老年人扶手杖可使支面扩大,重力作用线就不容易
        越出支面,稳度增大。
    \end{solution}
    \item 用载重汽车装运木箱,一些木箱装的是铁钉,另一些木箱装的是铝制器皿,怎祥装木箱,汽车的稳度比较大?


    \begin{solution}
        应该将装铁钉的木箱放在下面,装铝制器皿的木箱
        放在上面。这样整个车子重心较低、稳度较大。
    \end{solution}
    \item 下列情况是哪类平衡?
    \begin{enumerate}
        \item 体操运动员在吊环上做倒立;
        \item 体操运动员吊在吊环上;
        \item 杂技演员走钢丝;
        \item 轮子套在水平轴上.
    \end{enumerate}


    \begin{solution}
       (a)、(c)为不稳平衡。(b)为稳定平衡。(d)为随遇平衡。
    \end{solution}
    \item 把一个长方体立在一块木板上,在长方体前表面的中心点固定一个小螺钉,在小螺钉上拴一个小铅锤.把木块的一端慢慢抬起,研究一下长方体在什么条件下才倾倒?已知前表面的边长,你能否算出木块要抬起多大角度,长方体才倾倒?实际算一算,再用实验来检验.


    \begin{solution}
        这里所说的前表面是指面向读者的那一面。当小铅
        锤越出长方体前表面的底边时,长方体就要倾倒.图6.19画的是临界状态,从图上可以看出
\[\tan\theta=\frac{b}{a},\qquad \theta=\tan^{-1}\frac{b}{a}\]
        式中$a,b$分别为前表面的两个边长,若木板抬起的角度大于
        $\theta$, 长方体就要倾倒。
    \end{solution}
    \begin{figure}[htp]
        \centering    \includegraphics[scale=.6]{fig/6-19.png}
        \caption{}
        \end{figure}
\end{enumerate}











\subsection{习题}
\begin{enumerate}
    \item 略(课本已作解答)

\item 图6.23中的$AC$和$BC$是固定在电线杆上的铁棍,$AC$长0.9米,$BC$长1.2米,$AB$之间的距离是0.6米,在$C$处挂一盏重20牛的街灯,求$AC$对街灯的拉力和$BC$对街灯的支持力.

\begin{solution}
    设$AB$长为$a$, $BC$长为$b$, $AC$长为$c$. 则根据题意
    $a=0.6$米、$b=1.2$米,$c=0.9$米,对$c$点作受力分析,如图所
    示.$c$点受$AC$的拉力$F_1$, $BC$的支持力$F_2$和电灯的拉力$G$
    (等于电灯的重量)而平衡。根据共点力平衡条件,$F_1$和$F_2$的
    合力$F$应与$G$大小相等方向相反.从图中知,
    $\triangle FCF_2\backsim\triangle ABC$. 所以
\[\begin{split}
    \frac{F_1}{F}=\frac{c}{a},&\qquad F_1=\frac{c}{a}\x F=\frac{0.9}{0.6}\x 20=30{\rm N}\\
    \frac{F_2}{F}=\frac{b}{a},&\qquad F_2=\frac{b}{a}\x F=\frac{1.2}{0.6}\x 20=40{\rm N}\\
\end{split}\]
    所以$AC$对$C$点的拉力(也就是$AC$对电灯的拉力)$F_1$为30
    牛,$BC$对电灯的支持力$F_2$为40牛,方向如图.
\end{solution}

\item 略(课本已作解答)

\begin{figure}[htp]
\centering
\begin{minipage}[t]{0.48\textwidth}
\centering
\includegraphics[scale=.6]{fig/6-20.PNG}
\caption{}
\end{minipage}
\begin{minipage}[t]{0.48\textwidth}
\centering
\includegraphics[scale=.7]{fig/6-21.PNG}
\caption{}
\end{minipage}
\end{figure}



\item 一根木料,抬起它的右端要用480牛竖直向上的力,抬起它的左端要用650牛竖直向上的力,这根木料有多重?

\begin{solution}
图6.21反映了题意,抬起右端时,根据平衡条件,有
\[F_1\x AB-G\x AO=0\]
抬起左端时,有
\[G\x OB-F_2\x AB=0\]
两式相减,得
\[\begin{split}
    F_1\x AB+F_2\x AB-G(AO+OB)&=0\\
(F_1+F_2)\x AB&=G\x AB\\
G=F_1+F_2&=480+650=1130{\rm N}
\end{split}\]

说明:这一题中两次抬起时的支点$A$和$B$并不是实际
的固定转动轴,但同学们根据生活经验可以理解,在$F_1$、$F_2$刚
刚抬起一端时,木料是不会发生平动的,一定是绕另一端为轴
作转动。因此,仍可应用有固定转动轴物体的平衡条件。

\end{solution}
\item 放在斜面上的一个物体,当斜面倾角为$\theta$时,它沿着斜面匀速下滑.试证明物体和斜面之间的滑动摩擦系数$\mu=\tan \theta$.

\begin{solution}
物体受三个力作用:
重力$G$、斜面支持力$N$、滑动摩擦力$f$. 因为物体匀速滑下,所以满足共点力平衡条件。根
据平衡条件,$N$与$f$的合力$F$应与$G$大小相等方向相反,如
图6.22所示.在$\triangle OFf$中,
\[f=F\sin\theta =G\sin \theta,\qquad N=F\cos\theta=G\cos\theta\]
因为$f=\mu N$, 所以
\[\mu =\frac{f}{N}=\frac{G\sin \theta}{G\cos \theta}=\tan\theta\]
\end{solution}

\begin{figure}[htp]\centering
    \begin{minipage}[t]{0.48\textwidth}
    \centering
\includegraphics[scale=.7]{fig/6-22.png}
    \caption{}
    \end{minipage}
    \begin{minipage}[t]{0.48\textwidth}
    \centering
\includegraphics[scale=.6]{fig/6-23.png}
    \caption{}
    \end{minipage}
    \end{figure}

\item 如果天平的两臂并不完全相等,可以采取所谓复称法求得物体的真实质量$m$.先把物体放在左盘,砝码放在右盘,天平平衡时砝码的质量是$m_1$.然后把物体放在右盘,砝码放在左盘,天平平衡时砝码的质量是$m_2$.试证明物体的真实质量$m=\sqrt{m_1m_2}$.

\begin{solution}
    复称法的原理可用图6.23 表示,在第一种情况下,
    根据力矩平衡条件,得到$mg\ell_1=m_1g\ell_2$. 在第二种情况下,得到 $m_2g\ell_1=mg\ell_2$. 两式相除,得
\[\frac{m}{m_2}=\frac{m_1}{m}\]
所以
\[m^2=m_1m_2,\qquad m=\sqrt{m_1m_2}\]
\end{solution}
\item 无风的时候竖直下落的雨滴受到两个力的作用:重力、空气阻力.设空气阻力$f$与雨滴下落的速度$v$成正比,即$f=kv$.开始下落时雨滴的速度较小,空气阻力也较小,雨滴加速下落,速度随来越大.随着速度的增大,空气阻力也增大,当空气阻力增大到与重力平衡时,雨滴就以一定的速度匀速下落.这个速度叫做极限速度或收尾速度,设雨滴的质量为$m$,试证明雨滴的极限速度$v=mg/k$.


\begin{solution}
    当雨滴达到极限速度时,受力平衡,即$mg=f$. 根
    据题意,$f=kv$. 所以
    \[mg =kv,\qquad v=\frac{mg}{k}\]
\end{solution}
\item 用斜向上方的拉力拉
着一个物体在水平面上匀速前
进.已知拉力跟水平面成 $30^{\circ}$
角,拉力的大小是200牛,物体
的重量是500牛.求物体和水
平面间的滑动摩擦系数。

\begin{figure}[htp]
    \centering
\begin{tikzpicture}[>=latex,scale=1.3]
\fill[pattern=north east lines](-2,-.2) rectangle (2,0);
\draw (-2,0)--(2,0);
\draw (-.6,0) rectangle (.6,.8);
\draw[<->, thick](-1.5,.4)node[left]{$f$}--(1.5,.4)node[right]{$F_1$};
\draw[<->, thick](0,1.8)node[right]{$N$}--(0,.4)--(0,-1.3)node[right]{$G$};
\draw[dashed](1.5,.4)--(1.5,.4+.866)node[right]{$F$}--(0,.4+.866)node[left]{$F_2$};
\draw[->, thick](0,0)--(0,.4+.866);
\draw[->, thick](0,.4)--(1.5,.4+.866);
\draw(.4,.4) arc (0:30:.4)node[right]{$30^{\circ}$};

\end{tikzpicture}
    \caption{}
\end{figure}

\begin{solution}
    根据题意画出物体的
    受力图如图6.24所示,拉力$F=200$牛,与水平面成$30^{\circ}$角,物体的重量$G=500$牛,则拉力
    $F$在水平方向的分力
\[    F_1=F\cos 30^{\circ}=200\x\frac{\sqrt{3}}{2}=173{\rm N}\]

由于物体在水平方向上做匀速运动,所以摩擦力$f=F_1=173$牛.

拉力$F$在竖直方向上的分力
\[F_2=F\sin 30^{\circ}=200\x\frac{1}{2}=100{\rm N}\]
在竖直方向上,物体受到的水平面的支持力$N$、拉力的分
力$F_2$和重力$G$, 它们的合力为零,即$N+F_2=G$, 所以,
\[N=G-F_2=500-100=400{\rm N}\]
物体和水平面间的摩擦系数
\[\mu=\frac{f}{N}=\frac{173}{400}=0.43\]
\end{solution}
\item 图6.25是一台起重机的示意图,机身和平衡体的重量$G_1=4.2\times 10^5$牛,起重杆的重量$G_2=2.0\times 10^4$牛,其他数据如图中所示.起重机至多能提起多重的货物.
\begin{figure}[htp]
\centering\includegraphics[scale=.7]{fig/6-25.png}
\caption{}
\end{figure}
提示:这时起重机以$O$为转动轴而保持平衡.


\begin{solution}
    起重机的受力情况比较复杂,除图6.25中所画的
    三个力外,一般说还有两轮上受到的地面支持力。但当提起
    最大重量的货物时,左轮已差不多与地面脱离接触,全部重
    量压在右轮上,所以可以看成以$O$点为轴恰好力矩平衡,根
    据平衡条件,可列出
\[G_1\x\ell_1-G_2\x\ell_2-F\x\ell=0\]
式中$\ell_1$、$\ell_2$、$\ell$分别为$G_1$、$G_2$、$F$的力臂.所以
\[\begin{split}
    F&=\frac{1}{\ell}(G_1\x\ell_1-G_2\x\ell_2)\\
    &=\frac{1}{6}(42\x 10^4-6.0\x 10^4)=6.0\x 10^4{\rm N}
\end{split}\]
\end{solution}

\item  图6.26是一把杆秤,提扭和挂钩的距离$OB$是6.0厘米,秤锤的质量是1.2斤.不称物体时,把秤锤挂在$A$点,杆秤平衡,$A$点就是刻度的起点,设$OA$为1.6厘米,杆秤的质量为0.48斤,求杆秤的重心.在称某一物体时,秤锤移到$D$点后杆秤平衡,$AD$为24厘米,所称物体的质量是多少?
\begin{figure}[htp]
\centering\includegraphics[scale=.7]{fig/6-26.png}
\caption{}
\end{figure}

\begin{solution}
    当杆秤不称物体时,秤锤挂在$A$点而平衡,设$O'$为
    杆秤的重心,杆种重量为$G$, 秤锤的重量为$P$. 则根据力短平
    衡条件,
    \[P\x AO-G\x OO'=0\]
得:
\[OO'=\frac{P\x AO}{G}=\frac{1.2\x 1.6}{0.48}=4.0{\rm cm}\]
(此处重量$P$、$G$均以其质量的数值代入,下同)说明杆秤重
心在$O$点右侧4.0厘米处.
\begin{figure}[htp]
    \centering\includegraphics[scale=.7]{fig/6-27.png}
    \caption{}
    \end{figure}

当杆秤称某一物体时,设物体重$F$, 如图6.27所示。则
\[F\x BO-G\x OO'-P\x OD=0\]
\[\begin{split}
    F&=\frac{G\x OO'+ P\x OD}{BO}\\
    &=\frac{0.48\x 4.0+1.2\x (24-1.6)}{6.0}=4.8{\text{斤}}
\end{split}\]
被称物体的质量是4.8斤.
\end{solution}

\item 天平的横梁(连同指针)是一个有固定转动轴的物体,转动轴就是中央刀口$O$(图6.28).横梁的自重为$G$,重心$C$在指针上离转动轴$O$为$k$的地方.天平两盘中的重量相等时,作用在横梁两端的力$F$相等,横梁平衡,指针指在标尺的中央,即指针停在竖直方向,天平两盘中的重量稍有不等,横梁就要倾斜,指针随着偏移到较轻的一方,自重$G$对转动轴$O$的力矩将阻止横梁倾斜,最后横梁在某一倾斜位置上达到平衡.设指针与竖直方向成$\theta$角时横梁平衡,可以证明
\[\tan\theta =\frac{L}{Gh}(F_1-F_2) \]
从上式可以看出,对于天平两盘中一定的重量差,$G$和$h$越小,则$\theta$角越大,这种天平越灵敏.所以通常灵敏的天平要选用轻质材料做横梁,并要提高重心的高度.
\begin{figure}[htp]
\centering\includegraphics[scale=.7]{fig/6-28.png}
\caption{天平的平衡}
\end{figure}
试应用有固定转动轴的物体的平衡条件证明上式.

\begin{proof}
    从图6.28可以看出,当$F_1\ne F_2$时,重力$G$对$O$
    产生力矩,其力臂为$h\sin\theta$, 而$F_1$, $F_2$的力臂均为$L\cos\theta$. 根
    据有固定转动轴物体的平衡条件,有
\[F_2\x L\cos\theta+G\x h\sin\theta-F_1\x L\cos\theta=0\]
所以,
\[\tan\theta=\frac{L}{Gh}(F_1-F_2)\]
\end{proof}

\item  图6.29表示拖拉机(或汽车)在上坡.拖拉机前后轮的轮距是$L$,重心的高度是$h$,重心至前轮的距离是$\ell$.可以证明,上坡时不致向后翻倒,路面的最大倾角$\theta_1$满足条件:
\[\tan\theta_1=\frac{L-\ell}{h}\]
\begin{figure}[htp]
\centering\includegraphics[scale=.8]{fig/6-29.png}
\caption{}
\end{figure}
拖拉机下坡时,可以证明,下坡时不致向前翻倒,路面的最大倾角$\theta_2$满足条件:
\[\tan\theta_2=\frac{\ell}{h}\]

试证明上两式,并根据上两
式讨论:重心太高、太靠前、太靠后有什么不好?

\begin{proof}
    在拖拉机上坡时,路面倾角$\theta_1$过大,会使拖拉机
    向后翻倒。此时,实际上就是重力作用线已超过支面的后沿,
    即后轮与斜坡的接触点$B$, 而前轮则与斜坡脱离接触。此种
    情况如图6.30所示,从重心$C$作一条斜面的垂线相交于$D$
    点,则从$\triangle BCD$中可看出,
\[\tan\theta_1=\frac{L-\ell}{h}\]
\begin{figure}[htp]\centering
    \begin{minipage}[t]{0.48\textwidth}
    \centering
\includegraphics[scale=.7]{fig/6-30.png}
    \caption{}
    \end{minipage}
    \begin{minipage}[t]{0.48\textwidth}
    \centering
\includegraphics[scale=.7]{fig/6-31.png}
    \caption{}
    \end{minipage}
    \end{figure}

    在拖拉机下坡时,拖拉机容易向前翻倒,其将倒未倒的极
    限情况如图6.31所示,从$C$点作斜坡的垂线$CD$, 在$\triangle ACD$
    中可看
\[\tan\theta_2=\frac{\ell }{h}\]
从以上两式可以看出,如果重心太高即$h$太大,则$\tan\theta_1$
和$\tan\theta_2$就小$\theta_1$、$\theta_2$都小,说明无论上坡或下坡,都容易翻倒.
如果重心太靠前即$\ell$小,下坡时$\theta_2$就小,说明不宜在坡度较
大处下坡,如果重心太靠后即$(L-\ell)$小,$\theta_1$就小,说明不宜在
坡度较大处上坡。
\end{proof}
\end{enumerate}



\section{参考资料}
\subsection{静力学中物体受力的方向}
可以在空间作任意运动的物体叫自由体,如飞行中的飞
机,发射出去的炮弹等都是自由体。因受到周围物体阻碍、限
制而不能作任意运动的物体称为非自由体,限制非自由体运
动的其他物体,称为该非自由体的约束体 例如,摆动着的单
摆,摆绳是摆球的约束体,它限制摆球只能在不大于绳长的范
围内运动,而通常是在以绳长为半径的圆弧上运动。

促使物体运动(或有运动趋势)的力称为主动力 非自由
体在主动力作用下,将向某一方向运动,此如果受到约束体
的限制,非自由体将给约束体一个作用力,同时约束体给非自
由体一个反作用力。这个反作用力称为约束反力 因为约束
反力是限制物体运动的,所以它的作用点应在约束体与非自
由体相互连接或接触处。它的方向应与约束体所限制的运动
方向相反,这是确定约束反力方向和作用点的基本依据。

工程中常见的约束有以下几种基本类型,下面着重说明
约束反力的方向。

\subsubsection{柔性体约束}

由绳索、链条、胶带等形成的约束是柔
性体约束,这些约束体的特点是只能承受拉力,不能承受压力
和抗拒弯曲,因而只能限制物体沿着柔性体伸长的方向运动。
所以柔性体的约束反力只能是拉力,作用在连接点(或假想的
截断处),方向沿着柔性体的轴线而背离物体。课本中悬挂重
物的绳、牵引悬臂的钢素,对物体的作用力都属于这种情况。

\subsubsection{光滑接触面的约束}

在忽略摩擦,把接触表面作为
理想的光滑面的情况下,无论支承接触表面的形状如何,约束
体只能限制非自由体沿接触处公法线向接触体内的运动,不
能限制非自由体向任何其他方向的运动。所以光滑接触面的
约束反力只能是压力,作用在接触处,方向沿着接触表面在接
触处的公法线而指向物体。

\subsubsection{光滑圆柱形铰链约束}

圆柱形铰链是连接两个构
件的圆柱形零件,普通称为销钉 门窗上的合页,机器上的轴
承都属于圆柱形铰链。铰链约束只限制两个非自由体的相对
移动,而不能限制它们的相对转动。课本中塔式起重机的悬
臂与塔身的连接,三角形支架两支杆与墙壁的连接都作为铰
链连接处理,属于铰链约束,这种情况下的约束反力在垂直于
圆柱销轴线的平面内,通过圆柱销中心,方向(是压力或拉力)
视具体情况而定。

\subsection{力偶的性质}
力偶是两个具有特殊关系的力的组合,虽然力偶中的每
个力仍具有一般的力的性质,但在作为一个整体考虑它们对
刚体的作用时,则出现了与单个力不同的性质:
\begin{enumerate}
\item 力偶既没有合力,本身又不平衡,是一个基本的力
学量。
\item 力偶对于作用面内任一点之矩与矩心位置无关,恒
等于力偶矩。因此力偶对刚体的效应用力偶矩量度。
\item 作用在同一平面内的两个力偶,若其力偶矩大小相
等,转向相同,则该两个力偶彼此等效。这就是平面力偶的等
效定理。
\end{enumerate}

由等效定理可以得出两个推论:
\begin{enumerate}
\item 力偶可以在其作用面内任意转移,而不影响它对刚
体的效应。
\item 只要力偶矩大小和转向不变,可以任意改变力偶中
力的大小和相应地改变力偶臂的长短,而不影响它对刚体的
效应。
\end{enumerate}





