
\chapter{物体的平衡}
\section{教学要求}
这一章讲述静力学的基本知识,主要是讲在共点力作用
下物体的平衡和有固定转动轴的物体的平衡。

这一章的教学要求是:
\begin{enumerate}
\item 理解平衡和平衡条件的概念,掌握在共点力作用下物
体的平衡条件。
\item 理解力矩的概念,掌握有固定转动轴物体的平衡
条件。
\item 了解力偶和力偶矩的概念,知道力偶的作用是使物体
只发生转动。
\item 了解平衡的种类和稳度。
\end{enumerate}

下面对这一章的教学内容作些具体说明。

在动力学之后讲述静力学,有可能把静力学知识当成动
力学的特殊情况来理解,课本在实验的基础上得到共点力作
用下物体的平衡条件后,再从牛顿第二定律推导出这个平衡
分件,其目的就是要加深学生对平衡条件的理解。

讲述力对物体的转动作用,是为了讲解有固定转动轴的
物体的平衡做准备,这里讲述的转动的特点、转动快慢的描
述以及匀速转动和变速转动的概念,要求学生了解即可。关
于力矩,重点是要求学生理解力矩的概念,至于力矩的代数
和等于零或不等于零时物体怎样转动,使学生有个了解就可
以了。

关于有固定转动轴的物体的平衡条件的教学,还是以实
验为基础,但应注意实验与推理相配合,以加深学生的认识。

力偶是一个重要概念,而且以后讲通电线圈在磁场中运
动时会用到,因此应使学生对力偶知识有所了解。这里主要
是使学生明确知道力偶与一个力对物体的作用不同,--个力
可以使物体同时发生转动和平动,而力偶的作用是使物体只
发生转动而不发生平动。对有固定转动轴的物体来说,虽然
一个力的作用和力偶的作用都可以使物体发生转动,但效果
是有区别的,教材中说明这一点,是为了加强力偶的作用是
使物体只发生转动的认识。

关于平衡的种类和稳度的教学,都只要求作一般的介绍,
目的是扩展学生的知识面,把他们的静力学知识用来分析这
一常见现象。平衡的种类的区分,要让学生知道区分的标志
是以物体的平衡遭到破坏之后能否自行回到原来的平衡
位置。讲解稳度,要使学生知道在必要时增大稳度的方法。

理论联系实际,是物理教学的一条重要原则,静力学知
识在实际中很有意义,这一章的教学更应注意与生产和生活
实际的联系,所选习题力求有实际意义,而不要补充那些过难
而又缺少实际意义的题。

\section{教学建议}
本章教学建议分成四个单元,第一单元(全章引言和第
一节)主要讲述什么是物体的平衡状态和平衡条件,并得出
在共点力作用下物体的平衡条件。第二单元(第二、三节)
通过力对物体的转动作用的讨论,引入力矩这个重要概念,并
进一步得出具有固定转动轴的物体的平衡条件。第三单元
(第四节)讲述力偶的初步知识。第四单元(第五、六节)是对
物体平衡状态的进一步分析,说明物体平衡还有个稳定不稳
定的问题,稳定平衡还有个稳定程度的问题。

\subsection{第一单元}
本单元教学应该以实验为基础,做好三个共点力平衡的
演示实验。但同时要注意实验与推理相结合,这里主要是两
方面的推理:一是将三个共点力的平衡条件推广到三个以上
共点力作用下的平衡条件。虽然教材说“用实验还可以证明”,
但教学时一般不必再做三个以上共点力平衡的实验了。另一
个是从牛顿第二定律推出共点力作用下物体的平衡条件。

\subsubsection{对平衡状态的理解}

在全章引言中,教材明确指出:
“如果一个物体既不做平动,也不做转动,即保持静止,或者
做匀速直线运动或匀速转动,我们就说这个物体处于平衡状
态。”这是平衡状态的定义。比起以前教材中关于物体平衡的
说法,含义更广了。以前所说的平衡,仅指物体处于静止或匀
速直线运动状态。而这里的定义说,匀速转动状态也是平衡
状态。为什么说匀速转动也是平衡状态呢?因为它与静止和
匀速直线运动有共同之处,即运动状态保持不变。前者是物
体的即时速度保持不变(加速度为零),后者是物体转动的角
速度保持不变,这样,既使学生对平衡状态概念的理解进一步
深化,而且为利用牛顿第二定律推导出共点力平衡条件,也为
以后推导有固定转动轴的物体的平衡条件,作了准备,教材
中把物体的平衡安排在牛顿运动定律之后讲述,也为加深对
平衡这一概念的理解提供了条件。

\subsubsection{什么是共点力}

在教学中可以提一提,所谓共点力并
不一定是几个力都作用在同一点上,还应包括几个力的作用
线相交于一点的情况。因为作用在物体上的力沿着力的作用
线平移,其作用效果是相同的。此外还应注意,我们在这里所
讨论的共点力,仅限于在同一平面上的共点力,有的书上称为
共面力。如果不是共面力,情况就要复杂得多。

\subsubsection{三力平衡实验}

证明三个共点力作用下的平衡条件
是合力为零的实验并不难做,但要得出平衡条件则必须把实
验做得尽量准确一点.建议在教学中注意两点:
\begin{enumerate}
\item 课本图
6.1甲中的整个装置应是水平放置的.如果竖直放置,则可
能因下面一个弹簧倒置而产生较大误差。    
\item 如果教师演示
时,不便将装置水平放置,则可以将下面那个弹簧秤牵拉的力
改为挂上砝码,当然,受力物体应该选择重量很小的物体,或
者就可以把绳子的结点作为受力平衡的研究对象。
\end{enumerate}

\subsubsection{平衡条件的应用}

共点力的平衡条件$F_{\text{合}}=0$原是一
个矢量方程,由于教材不介绍正交分解法,所以必须设法将
平衡条件简化,例如本章后习题1的解答中,先将$F_1$和$F_2$
合成为$F'$, 接着就把方程$F_{\text{合}}=0$简化为$F'$与$F$大小相等、方
向相反的结论,再通过几何关系分别求出$F_1$和$F_2$. 如何将
平衡条件$F_{\text{合}}=0$简化成既符合题意,又便于研究的形式,是学
生学习静力学时常常感到困难的问题,教学中要有意识加以
指导。

\subsubsection{静力学问题宜用平衡条件解}

学生在第一章学习力
的分解时,实际也解过类似本章后习题1的题目.所以学生
可能习惯于用力的分解来解题。在比较简单的情况下,从力
的作用效果出发,用力的分解来解题也是可以的,例如本章后
第1题,但如果题目复杂一些,用力的分解来解则可能发生
差错。因为有时某个力在某方向上的分力是没有实际意义的。
所以学了这一单元以后,建议学生以后可尽量用平衡条件来
解静力学问题。


\subsection{第二单元}
本单元讲述力对物体的转动作用、力矩的概念和有固定
转动轴物体的平衡条件。
\subsubsection{物体转动与物体平动的对应关系}

物体转动与质点
运动虽然是两种不同的运动,但它们的许多概念和规律都有
类似的关系,可在适当的时候进行对照比较。例如课本中所
说的,平动物体上各点的速度都相同,任何一点的速度可以
代表整个物体的速度。同样,转动物体上各点的角速度都相
同,任何一点的角速度都可以代表整个物体的角速度。这样,
平动物体的速度与转动物体的角速度相对应,另外还有,平动
的位移和转动时转过的角度(称为角位移)相对应,匀速直线
运动与匀速转动相对应,共点力的平衡条件与力矩平衡条件
相对应,等等。教学中教师可以启发学生进行这样的对比,有
利于学生对转动知识的理解。

\subsubsection{力矩的概念和计算}
学习力矩的概念,关键在于掌握
力臂的概念。关于力臂的意义和计算,虽然在初中学习杠杆
平衡条件时已经反复练习过,但在高中阶段学生还是常常容
易搞错。所以这里仍要多举几个例子让学生学会确定力臂的
长度。在此基础上,高中还要学生掌握力矩的单位和正负。
力矩的单位是${\rm N\cdot m}$,不是${\rm J}$,要求学生不要同功的单位混
淆。力矩有正负,但在教学中不要提力矩的方向。因为力矩
方向的规定涉及矢量乘法,已超出高中物理范围。

\subsubsection{关于有固定转动轴物体平衡的实验和推理}

通过实验
和推理得出有固定转动轴物体的平衡条件是本单元的重点。
课本是先通过推理得出平衡条件,然后用实验来验证的。教
学中如果教师采用先实验再说理的办法也是可以的。经过实
验,学生对力臂的计算和有固定转动轴物体的平衡条件将有
较深刻的理解.在将结论$M_1+M_2=M_3$改写成$M_{\text{合}}=0$的过
程中,教师还得作一些解释。前者等号两侧都是绝对值,等号
左侧为使物体向顺时针方向转动的力矩,右侧为使物体向逆
时针方向转动的力矩。后者$M_{\text{合}}$中包括了正负两种力矩。写
成$M_{\text{合}}=0$的形式便于同共点力平衡条件$F_{\text{合}}=0$对照.$F_{\text{合}}=0$
为一矢量方程,而$M_{\text{合}}=0$中的力矩不是正,就是负,是个代数
方程.因此在运用上$M_{\text{合}}=0$要比$F_{\text{合}}=0$容易掌握一些。

\subsubsection{关于固定转动轴的理解}

课本在对固定转动轴的理
解上作了一些扩展,即不限于研究确有实际的固定转轴的情
况,而把在研究问题时该物体绕某线转动的那条线看作是固
定转动轴,本章末习题第4题和第9题就是这样,解这类
题可以培养学生灵活应用所学知识的能力,但课本对一般平
面力作用下物体平衡问题,即须同时使用平动平衡条件和转
动平衡条件的问题一概不作要求,教师应加以注意。

\subsubsection{三角支架问题在什么情况下要用力矩平衡来解}

在支架本身重量可以忽略不计的情况下,三角支架问题可以用
共点力平衡条件来解,也可以用具有固定转动轴的物体的平
衡条件来解,但当支架本身重量不能忽略时,如本章后习题
第3题,则必须以横梁$BO$为研究对象,以$B$为转动轴,利用
力矩平衡条件来解.如果此时我们仍以$O$点为研究对象,$O$
点除受钢绳$AO$的拉力$T$和重物的拉力$F$以外,一定还有一
个$BO$对$O$点的支持力$N$. $T$、$F$、$N$三个力是共点的,$O$点又处
于静止状态,因此这三个力也一定满足$F_{\text{合}}=0$这个条件,但
是,当$BO$梁的自重不能忽略时,它对$O$点的支撑力$N$不再
沿着$BO$的方向.$N$的方向未定,用$F_{\text{合}}=0$来解,条件就不充
足了,如果我们用有固定转动轴的平衡条件先求出$T$的大小,
然后再利用共点力平衡条件可以求得$BO$对$O$点的支撑力
$N$, 很容易看出这个力不在$BO$的方向上。对程度较好的学
生,启发他们作以上的分析,对培养学生分析问题的能力很有
好处。


\subsection{第三单元}
这一单元介绍力偶这个概念,教学中应突出两个主要问
题:第一是使学生掌握力偶、力偶臂、力偶矩的意义,力偶矩大
小的计算和正负的规定;第二是要使学生明确,力偶的作用
是使物体只发生转动而不发生平动。
\subsubsection{关于力偶的作用}

力偶的作用是使物体只发生转动
而不发生平动,这个结论是通过力偶对物体的作用和一个力
对物体的作用的对比实验来使学生认识的.课本中图6.11和
图6.12的实验可利用书写幻灯进行投影增大可见度,被拉
动的物体可用一块圆形的有机玻璃板,上面画一些辐条,这
样在书写幻灯的投影下,可以清楚地看到平动和转动。课本
第三段说到用一只手板套筒容易磨损螺纹,用两只手板套筒,
轴上就不会受到压力,目的也是要突出力偶只使物体发生转
动.这一点,也可以利用课本图6.11和图6.12的装置加以
说明。具体做法见本章实验指导。

\subsubsection{力偶矩也是力矩}

教学中要使学生明白,力偶矩实际
上就是力偶的两个力的力矩之和。力偶矩和力矩的作用都是
使物体的转动状态发生变化,因此,不要把力偶矩与力矩这
两个概念孤立开来。教学中,要通过力矩代数和的计算推导出
力偶矩$M=Fd$的关系式。通过推导,还要启发学生理解力
偶矩的大小不随所取的转动轴位置的改变而改变。至于在力
偶的作用下物体究竟以哪一点为轴转动的问题,课本没有提,
教学中也应回避。


\subsection{第四单元}
本单元对物体平衡的问题作进一步的分析,目的是扩展
学生的知识面,把物理知识与实际联系得更紧密。教师在教
学过程中,应多举些实例,多做些演示,以加深对教材的理解,
培养学生观察和思考问题的能力。

本单元的重点是要学会怎样分析稳定平衡、不稳平衡和
随遇平衡,而不是光记忆这些结论,对有支点的物体来说,是
根据它偏离平衡位置后,重力和支持力的合力是否能
使物体回到平衡位置来判断的;对有支轴的物体来说,是
根据它偏离平衡位置后,合力矩能否使物体回到平衡位置来
判断的。或者两者都用重心的升高或降低来判断。教师在分
析各种实例时,都要从这个角度来分析。

在所举的实例和演示中,既要注意简単叨了,便于分析,
联系实际,也要注意生动活泼,活跃课堂气氛,例如不倒翁的
例子就比较生动。此外,杂技团演员走钢丝、顶碗等实例都有
个平衡和稳定问题。有条件的学校,可以放映一些科技电影,
如“杂技团的秘密”等。

\section{实验指导}
\subsection{演示实验}
\subsubsection{共点力作用下物体的平衡条件}
课本图6.1所示的演示实验是在水平面上做的,可
以通过投影仪显示,物体可用
有机玻璃做,这样,可以看到作
用点。

\begin{figure}[htp]
    \centering
   % \includegraphics[scale=.7]{fig/6-1.png}
    \caption{}
\end{figure}

还可以利用两个弹簧
秤和一组钩码来进行演示,如
图6.1所示,要使学生认识这
个实验中的研究对象是绳的结
点$O$(也可以在$O$点固定一块用泡沫塑料锯成任意形状的薄
板作为象征性的物体),$O$点受三个共点力的作用而保持平
衡。如果调节好两个弹簧秤的位置,使得$F_1=120$克力.
$F_2=90$克力,$F_1$和$F_2$的夹角为$90^{\circ}$, 所用的钩码的总重量为150
克力。就可以方便地得出在共点力作用下,物体的平衡条件
是合力等于零。

如图6.2所示,一辆小车放在光滑斜面上,小车一端
通过细绳和弹簧秤相连,弹簧秤的另一端固定在斜面顶端,当
小车静止时,小车受到三个力而平衡,使斜面倾角改变,可以
看到弹簧秤的示数也发生相应的变化,达到某一数值时,小车
又处于平衡状态。

\begin{figure}[htp]
    \centering
   % \includegraphics[scale=.7]{fig/6-2.png}
    \caption{}
\end{figure}

\subsubsection{力矩的作用}

\begin{figure}[htp]
    \centering
   % \includegraphics[scale=.7]{fig/6-3.png}
    \caption{}
\end{figure}

可以利用教室的门做如图6.3所示的演示,把橡皮绳的
一端系在门把手上,另一端捏在手中,先将门关上,沿着垂
直于门面的方向拉动橡皮绳,观察到橡皮绳稍有伸长,门就
能被拉开(如图中橡皮绳的拉力为$F_1$);然后重新把门关上,
改变拉像皮绳的方向,使得拉力仍沿水平方向但不与门面
垂直。这时可观察到橡皮绳必须拉得更长,才能使门拉开(如
图中橡皮绳的拉力增大为$F_2$)。启发学生思考,为什么会出
现这种现象?说明什么问题?然后演示橡皮绳在水平方向沿着
门面拉动,直到橡皮绳被拉断,还是拉不开门,让学生思考这
又是为什么?

\subsubsection{有固定转动轴的物体的平衡条件}
正力矩和负力矩的规定:
根据课本图6.8, 用力矩盘进行演示,指出力$F_1$和$F_2$所产
生的力矩是负的,它们的作用效果是使有固定转动轴的物体
向顺时针方向转动、力$F_3$产生的力矩是正的,它的作用效果
是使物体向反时针方向转动。

仍利用课本图6.8所示的力矩盘进行演示,并可改
变细绳下端所挂钩码的个数,改变作用点的位置。当力矩盘
平衡时,从各个力和力臂的大小可以得出:有固定转动轴的物
体的平衡条件是力矩的代数和等于零。

\subsubsection{应用有固定转动轴的物体的平衡条件解题}
课本213页习题第4题,可用模拟演示来加以说
明,如图6.4所示,将一根粗细
不均匀的木棍(如教棒),平放
在讲台上,在棍的两端各系一个细绳套,用弹簧秤先后勾住
棍的细端和粗端的绳套,稍稍提起,使该端脱离桌面,分别读
得弹簧秤的示数为$F_1$和$F_2$. 再用弹簧秤勾住绳套,把木棍整个
提起来,读出弹簧秤示数为$F$。从实验结果可知:$F=F_1+F_2$


\begin{figure}[htp]
    \centering
   % \includegraphics[scale=.7]{fig/6-4.png}
    \caption{}
\end{figure}

课本213页第9题可用铁架台及顶部装有定滑轮
的木杆制成模型进行演示(图6.5)。


\begin{figure}[htp]
    \centering
   % \includegraphics[scale=.7]{fig/6-5.png}
    \caption{}
\end{figure}

当在定滑轮下所挂的重物
$G$的重量过大时,铁架台将向
右倾倒,可以减小重物$G$的重
量或缩短拉紧撑杆的细绳长
度,使得撑杆的倾角变大,直到
铁架台恰巧不倾倒。应该指出,
在这个模拟演示中是将整个铁
架台连同撑杆和所挂重物看成
一个物体来进行研究的,课本213页第9题的起重机也应
该看作是一个整体,各个力对前轮$O$所产生的力矩代数和应
等于零,来求出起重机至多能提起多重的物体。

\begin{figure}[htp]
    \centering
   % \includegraphics[scale=.7]{fig/6-6.png}
    \caption{}
\end{figure}

如图6.6所示,将均
匀米尺的一部分伸出水平桌面
外,在米尺伸出部分的顶端放
一小砝码(10克或20克),调节
米尺伸出部分的长度,使得米尺仅对桌边有压力,这时就可以
把米尺跟桌边接触的地方看成是固定转动轴。只要读出伸出
部分的米尺长度,计算从均米尺的重心位置到桌边的距离,
从已知砝码的重量,根据有固定转动轴物体的平衡条件,即可
求出米尺的重量。

\subsubsection{力偶}
课本图6.11和图6.12的演示可以通过投影仪来显
示,圆盘可以用厚一些的有机玻璃板来做,在圆盘的侧边车制
一凹槽,以便绕线。为了使效果明显,在有机玻璃圆盘上沿着
半径方向可用透明漆画几条有色条纹。

\begin{figure}[htp]
    \centering
   % \includegraphics[scale=.7]{fig/6-7.png}
    \caption{}
\end{figure}

如图6.7所示,在铁架台上通过复夹和试管夹安装
一块水平放置的中央有孔的有
机玻璃板,在平板上放一个准
备好的有机玻璃圆盘。在盘心
位置开一个圆孔,将一细竹针
穿过圆孔,并用复夹使竹针的
两端固定。然后在圆盘的一侧
通过缠绕的细线拉圆盘,可以
观察到圆盘转动的同时,竹针弯曲,这说明力矩的作用可以使有固定转动轴的物体发生转
动,但转动轴是受力作用的。如果不存在固定转动轴(将竹针
抽去),在力矩的作用下圆盘将同时发生转动和平动。如果对
圆盘作用一个力偶则可观察到圆盘转动时竹针并不弯曲,说
明竹针并不受力。因此即使抽去竹针,在力偶作用下,圆盘也
只发生转动,不发生平动。

\subsubsection{物体平衡的种类}
课本图6.16有支点的物体的平衡的演示,可利用小球放
在离心轨道(间距小于小球直径的两根平行铁丝)上来进行演
示.课本图6.17有支轴的物体的平衡,可用一均匀薄木板
(厚度约为2—3mm)来演示,中间的小孔要开在薄木板的重
心上,孔内侧要粗糙些,这样,演示随遇平衡的效果会好些。


\subsubsection{稳度跟重心的高低和支面的大小有关}
可制成一个如图6.8所示的形状可变的框架(稳度演示
器)来进行演示,当把它由长
方体改变成斜方体时,只要系
在它的中心(表示重心位置)的
重垂线不超出支面,则斜方体
就不会倾倒。
如果重垂线超出
支面,斜方体就会倾倒。

\begin{figure}[htp]
    \centering
   % \includegraphics[scale=.7]{fig/6-8.png}
    \caption{}
\end{figure}

\subsection{学生实验}
\subsubsection{研究有固定转动轴物体的平衡条件}

实验时,先用胶纸在力矩盘上粘贴一张白纸,用手指
隔着白纸在转动轴部位按一下,在纸上留下转动轴的痕迹,然
后按课本图10.15所示的装置把力矩盘装好.要注意转动轴
应在承平方向,使力矩盘位于竖直平面内。在盘上任意选择四
个位置,各插一根大头针(要插深些),再按课本的要求(课本图10.15)在三根针上用细线悬挂钩码,悬挂细线时要造当
靠近大头针的根部,但又不要使细线和盘面发生摩擦。在第
四根针上用细线钩在弹簧秤的钩上,要注意调节固定在横杆
上的弹簧秤的位置,使得力矩盘平衡时,弹簧秤的拉线不要通
过转动轴。

当力矩盘在这四个力作用下处于平衡状态时,记下
这四个力的大小。用削细的铅笔沿着四根细线的方向,在离开
大头针较远的地方,分别画上一个“$\x$”号,并在悬挂这些细线
的大头针的针孔周围画一小圆,以便确定这四根拉力的方向。

取下钩码和弹簧秤,拔去大头针,取下白纸;用直尺
将做过记号的针孔和相关的“$\x$”号用虚线连接起来,并根据
自己选定的标度(要在记录纸上明确标出),按力的图示法分
别画出这四个拉力$F_1$、$F_2$、$F_3$和$F_4$.

根据事先在白纸上所做的记号,画出固定转动轴$O$,
然后分别作出从$O$点到四个拉力作用线的垂线,并用毫米刻
度尺量出这些垂线的长度,这就是力臂,在原始记录纸上标出
各个力臂$L_1$、$L_2$、$L_3$和$L_4$.

这个实验中也可以不
用横杆来固定弹簧秤,而用另
一个铁架台,通过复夹用试管
夹把弹簧秤背面的铁壳夹紧
(图6.9),这样做的好处在于可
以方便地调节弹簧拉长的方
向,使它跟细线的方向一致。

启发学生讨论课本最后提出的问题,认识到这是为
了便于使力矩盘平衡。因为弹簧秤在它的量程范围内,拉力的。
大小可以连续变化,这样就可以自行改变弹簧秤拉力的大小,
使得力矩盘平衡。

\subsection{课外实验活动}
\subsubsection{制作杆秤}
在练习制作时,可注意并思考以下几点:
\begin{enumerate}
\item 制作杆秤所用的细木棍要挑比较结实一些的材料,但
不要太粗,如果找不到合适的材料,找一根较长的竹筷或木筷
也可以。提纽要用强度大一些的细绳,秤钩可以用粗铁丝制
成,(也可以用几股细铁丝绞起来做),秤锤用质量小于1千克
的物体也可以。
\item 在秤钩不挂物体的情况下,把秤锤挂在秤杆上确定
秤杆的零刻度(课本图6.26的$A$点)时,要注意秤杆
和秤钩(看成一体)的重心在提纽$O$点的哪一侧?
上
\item 证明秤杆上刻度间的距离为什么是均匀的。可以结
合课本214页第10题的计算进行讨论.
\item 要增大杆秤的称量范围,可以再装一个提纽(称做二
纽),想想看这个二纽的位置应该离秤钩远一些还是近一些?
\end{enumerate}









































\section{参考资料}
\subsection{静力学中物体受力的方向}
可以在空间作任意运动的物体叫自由体,如飞行中的飞
机,发射出去的炮弹等都是自由体。因受到周围物体阻碍、限
制而不能作任意运动的物体称为非自由体,限制非自由体运
动的其他物体,称为该非自由体的约束体 例如,摆动着的单
摆,摆绳是摆球的约束体,它限制摆球只能在不大于绳长的范
围内运动,而通常是在以绳长为半径的圆弧上运动。

促使物体运动(或有运动趋势)的力称为主动力 非自由
体在主动力作用下,将向某一方向运动,此如果受到约束体
的限制,非自由体将给约束体一个作用力,同时约束体给非自
由体一个反作用力。这个反作用力称为约束反力 因为约束
反力是限制物体运动的,所以它的作用点应在约束体与非自
由体相互连接或接触处。它的方向应与约束体所限制的运动
方向相反,这是确定约束反力方向和作用点的基本依据。

工程中常见的约束有以下几种基本类型,下面着重说明
约束反力的方向。

\subsubsection{柔性体约束}

由绳索、链条、胶带等形成的约束是柔
性体约束,这些约束体的特点是只能承受拉力,不能承受压力
和抗拒弯曲,因而只能限制物体沿着柔性体伸长的方向运动。
。所以柔性体的约束反力只能是拉力,作用在连接点(或假想的
截断处),方向沿着柔性体的轴线而背离物体。课本中悬挂重
物的绳、牵引悬臂的钢素,对物体的作用力都属于这种情况。

\subsubsection{光滑接触面的约束}

在忽略摩擦,把接触表面作为
理想的光滑面的情况下,无论支承接触表面的形状如何,约束
体只能限制非自由体沿接触处公法线向接触体内的运动,不
能限制非自由体向任何其他方向的运动。所以光滑接触面的
约束反力只能是压力,作用在接触处,方向沿着接触表面在接
触处的公法线而指向物体。

\subsubsection{光滑圆柱形铰链约束}

圆柱形铰链是连接两个构
件的圆柱形零件,普通称为销钉 门窗上的合页,机器上的轴
承都属于圆柱形铰链。铰链约束只限制两个非自由体的相对
移动,而不能限制它们的相对转动。课本中塔式起重机的悬
臂与塔身的连接,三角形支架两支杆与墙壁的连接都作为铰
链连接处理,属于铰链约束,这种情况下的约束反力在垂直于
圆柱销轴线的平面内,通过圆柱销中心,方向(是压力或拉力)
视具体情况而定。

\subsection{力偶的性质}
力偶是两个具有特殊关系的力的组合,虽然力偶中的每
个力仍具有一般的力的性质,但在作为一个整体考虑它们对
刚体的作用时,则出现了与单个力不同的性质:
\begin{enumerate}
\item 力偶既没有合力,本身又不平衡,是一个基本的力
学量。
\item 力偶对于作用面内任一点之矩与矩心位置无关,恒
等于力偶矩。因此力偶对刚体的效应用力偶矩量度。
\item 作用在同一平面内的两个力偶,若其力偶矩大小相
等,转向相同,则该两个力偶彼此等效。这就是平面力偶的等
效定理。
\end{enumerate}

由等效定理可以得出两个推论:
\begin{enumerate}
\item 力偶可以在其作用面内任意转移,而不影响它对刚
体的效应。
\item 只要力偶矩大小和转向不变,可以任意改变力偶中
力的大小和相应地改变力偶臂的长短,而不影响它对刚体的
效应。
\end{enumerate}





