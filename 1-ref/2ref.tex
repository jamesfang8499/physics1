\chapter{直线运动}
\section{教学要求}
这一章讲授的运动学知识,跟第一章一样,都是基础性
的,是后面学习动力学所必需的预备知识。

为了减少学生学习的困难,适应学生的知识水平和接受
能力,本章只讲直线运动,而把运动的合成和分解以及平抛和
斜抛的知识移到第四章的曲线运动中去讲。

通过这一章的教学,应该使学生了解一些描述物体运动
的基本概念和方法,掌握匀速直线运动和匀变速直线运动的
规律,会用这些规律来分析解决一些比较简单的实际问题。

这一章的教学要求是:
\begin{enumerate}
\item 了解参照物的概念,知道研究物体的运动要选择合适
的参照物,了解质点的概念,知道在什么情况可以把物体看
成质点。了解位移的概念,知道位移和路程的区别。
\item 明确什么是匀速直线运动,掌握匀速直线运动的公
,理解匀速直线运动的图象的物理意义。
\item 明确什么是匀变速直线运动,理解平均速度、即时速
,加速度等概念,明确知道速度和加速度的区别,掌握匀变
直线运动的公式,理解匀变速直线运动的图象的物理意义。
\item 认识自由落体运动和竖直上抛运动的特点和规律.
\end{enumerate}

下面对这一章的教学内容作些具体说明。

第一节开始先复习初中学过的机械运动和参照物的
概念,以加强与初中知识的联系,同时强调参照物的重要性,
使学生初步了解怎样选择参照物。第一节还简单介绍了平动
和转动,目的是使学生对物体运动的这两种基本形式有所认
识,后面用到时方便。教材没有给平动和转动下严格的定义,
也不要求补充讲解,只要求学生知道平动和转动的特点和
区别。

质点是力学中的一个重要概念,它是通过科学抽象得出
的理想化模型,运用理想化模型来研究问题,是物理学经常
用的方法。学生在这里初次接触这个问题,需要引起他们注
意,因此把质点单独作为一节来讲述。关于质点的定义,教材
采用了有质量的点这种说法,目的在于强调质点是物理学上
的点,不同于几何学中所说的点,在“质点”这节的最后,给出
了研究质点运动的基本线索——确定质点在任一时刻的位置
和速度,是为了使学生明确讲解本章知识的思路。

描述物体的运动,首先要懂得如何描述物体的位置和位
置变化,为要讲解位置的坐标表示和位移的概念。对位移
的坐标表示,只要求学生知道位移的数值可以用初末位置的
坐标来表示;在后面计算位移的公式中,除了平抛和斜抛外,
都不要求写出位移的坐标表示,而只写出位移本身。

匀速直线运动的知识,学生在初中已学过。这里要在复
习的基础上予以扩展和提高,为讲授匀变速直线运动作好准
备。用比值来定义物理量是物理学中常用的方法,这里用位
移和时间的比值重新定义了速度。在用比值给出速度的定义
之后,说明速度在数值上等于单位时间内位移的大小,使学生
