
\chapter{运动定律}
\section{教学要求}
这一章学习动力学,讨论运动和力的关系。牛顿运动定
律是动力学的基础,这一章学习的中心内容就是牛顿运动
定律。

这一章的教学要求是:
\begin{enumerate}

\item 掌握牛顿第一定律,明确力、惯性等概念的物理意义.
\item 正确理解和掌握牛顿第二定律;掌握力学单位制;了
解超重和失重。
\item 会运用运动学公式、力的合成和分解以及牛顿运动定
律分析解决综合性问题。
\end{enumerate}

下面对这一章的教学内容作些具体说明。

运动和力的关系问题是动力学的基本问题,要建立正确
的认识又是颇不容易的,学生从日常经验出发,往往产生错
误认识,有的认识同历史上前人产生过的错误有类似之处。在
讲述牛顿第一定律之前,介绍人类对这个问题的认识过程,可
以帮助学生理解和掌握运动和力的关系。课本里介绍了伽利
略的理想实验,使学生知道伽利略是怎样得出正确结论的,并
了解在实验事实的基础进行科学思维的研究方法,以培养他
们的思维能力。

为了使教学从牛顿第一定律比较自然地过渡到牛顿第二
定律,在讲过牛顿第一定律之后,用一节教材讲述怎样使物体
的运动状态发生改变。这里,把力和运动联系起来,指出力是
使物体产生加速度的原因,加深学生对力这个概念的理解。这
一节,还讲述了质量与运动状态改变的关系,明确质量在动力
学中的意义,从而为讲述牛顿第二定律做好准备。

牛顿第二定律是这一章的重点内容。采用在教师指导下
由学生自己动手做实验,最后得出结论的做法来讲述牛顿第
二定律,可以调动学生的学习积极性和主动性,培养学生通过
实验研究问题的能力,并使学生对牛顿第二定律有深刻的
印象。

在讲述加速度和力的关系时,不可忽略它们的方向总是
一致的,这对后面研究曲线运动很重要。

以后学习匀速圆周运动、简谐振动等部分时,要学到变力
和变加速度的概念。有必要指出牛顿第二定律不仅适用于恒
力作用下的运动,而且适用于变力作用下的运动;外力随时间
改变时,加速度也随时间改变。但不要求讲述即时加速度这
个概念。

讲述力的独立作用原理,可以使学生进一步掌握在有几
个力作用时如何应用牛顿第二定律。这里,要紧的是懂得加
速度的大小和方向是由合外力决定的。

牛顿第二定律的应用有两个方面:一是已知物体的受力
情况,确定物体的运动情况;二是已知物体的运动情况,确
定物体的受力情况。利用动力学知识,知道了力和加速度,
也可以确定物体的质量。习题中带解的题目,可以使学生对
用动力学方法确定质量有所了解。

在演示实验的基础上讲述超重和失重时,涉及到弹簧秤
和被测物体组成的一个连接体。这里,只要明确研究对象是被
测物体,分析这个物体的受力情况,列出方程,再应用牛顿第
三定律,问题也就解决了。在这里既不必分别列出弹簧秤和
被测物体的动力学方程,更不要系统地讲解连解体问题。通过
超重和失重这一节的学习,还应使学生知道,某些动力学问题
常常需要综合运用牛顿第二定律和牛顿第三定律才能解决。

关于牛顿运动定律的适用范围,应要求学生有初步的了
解。限于学生的水平,这个问题不可能作深入的讨论,对于刚
体、相对论、量子力学等名词,可以只稍加说明,也不要求学生
记住这些名词。

这一章的练习和习题是按照循序渐进的原则,由浅入深,
由易到难来安排的。具体的安排是:在得出$a\propto F$和$a\propto \dfrac{1}{m}$两个公式后,安排了仅限于应用上述正比、反比关系的题目;在
得出牛顿第二定律的公式$F_{\text{合}}=ma$之后,安排了由几个力的
作用使物体产生加速度的题目;在“力学单位制”一节后面,安
排了物体受一个力作用时用动力学和运动学的知识求解的题
目;在讲完全章内容后,安排了在多个力作用下用动力学和运
动学知识求解的较复杂的综合性题目。

\section{教学建议}
这一章可分三个单元进行教学。第一单元包括引言和第
一节《牛顿第一定律》;第二单元从第二节《物体运动状态的改
变》到第七节《力学单位制》;第三单元从第八节《牛顿运动定
律的应用(一)》到第十一节《牛顿运动定律的适用范围》。

\subsection{第一单元}
在这一单元的教学中,首先可以指导学生认真阅读“历史
的回顾”这段课文,认识牛顿第一定律是怎样在伽利略理想实
验的基础上总结出来的。

\subsubsection{力不是维持运动的原因}

这一内容的教学,可以从学
生的生活经验提出所要研究的问题。譬如为了使足球不致停
下来,运动员带球前进时必须不断用脚轻轻地拨动球;又如为
了自行车不致减慢速度,总要不断地用力蹬脚踏板。让学生
带着这些问题来阅读“历史的回顾”这一段课文,然后演示教
材上图3.1的斜面对接实验(可用粗铁丝弯制成导轨来代
替对接的斜面).在演示图3.1丙的实验时,也可以使一辆
小车从斜面的一定高度上下滑后,在水平面上运动,让学生观
察在阻力比较大的水平面上,小车速度减慢得比较快,在阻力
不很大的平面上,小车速度减慢得比较慢,从而认识小车在
平面上运动所以会减慢速度,恰是由于阻力的作用,阻力的
大小不同使小车速度变慢的快慢程度也不同。同样,足球在
草地上滚动速度所以会变慢,不用力蹬脚踏板自行车所以会
减慢速度,也说明有阻力的作用,因此力不是维持运动的原
因,而是改变物体运动状态的原因。

讲清这个问题,还可以使学生体会到来自生活经验的直
观感觉不一定都是正确的,要学会用科学的思想方法去分析
现象,掌握事物发展变化的规律。

在教学中要引导学生认识伽利略的理想实验是以可靠事
实为基础的科学推论。可引导学生思考,刚才所做的演示实
验中如果水平面是光滑的,那么小车在水平面上运动时,它的
运动状态是否还会发生变化呢?小车将做什么运动?接着可以
进一步演示,使小车从斜面上滑下后在一块玻璃板上运动,可
以观察到小车运动的速度几乎没有变化,以加深对伽利略理
想实验的事实依据的认识。

\subsubsection{惯性和牛顿第一定律}

要引导学生对惯性的概念有
正确的理解。有的学生认为只有运动的物体才具有惯性;也
有的学生认为只有静止的物体才具有惯性,这些看法都是不
正确的。要使他们认识惯性是物体的固有属性,它和物体的
运动状态无关;但在力的作用下,物体运动状态改变的快慢
程度却与惯性有着直接的关系(牛顿第二定律所要研究的内
容)。而牛顿第一定律则阐明了物体的匀速直线运动状态或
静止状态都不需要力来维持,外力的作用只是改变物体的运
动状态,并不能改变物体的惯性。

\subsubsection{牛顿第一定律描述的是一种理想化的状态}

不受外
力作用的物体是不存在的。要使学生认识,即使前面所做的
演示实验中,当小车在光滑的水平面上运动时,不受任何阻
力,小车做匀速直线运动,仍然不属于牛顿第一定律所描述的
理想化状态。因为小车在水平面上运动时还受到重力和水平
面的支持力的作用,只是在水平方向上不受力,所以小车在水
平方向上的运动状态并不发生改变,牛顿第一定律所描述的
是一种理想化的状态,不同于可以用实验直接验证的牛顿第
二定律。

\subsection{第二单元}
这一单元以讲解牛顿第二定律为中心,是本章的重点。
\subsubsection{物体运动状态的改变}
教学中应该让学生认识,物体的运动状态决定于物体的
速度,根据牛顿第一定律可知,物体运动状态发生改变,必定
是外力作用的结果,为了帮助学生加深理解,可以举这样的
例子:要使静止在粗糙水平面上的物体开始运动,必须有外力
的作用,而且这个外力必须等于、大于物体与水平面间的最大
静摩擦力,物体才能改变它的运动状态,从静止开始运动;接
着,如果将所加的外力撤去,则物体只受到跟运动方向相反的
滑动摩擦力的作用,所以立即又改变它的运动状态,使运动速
度逐渐减小直到停止运动。


\subsubsection{牛顿第二定律}

为了加深对这一重要定律的理解,培
养学生用实验手段研究问题的能力,教材安排先由学生进行
实验,再由实验结果得出牛顿第二定律,在这个基础上,应使
学生认识以下几点:
\begin{enumerate}
    \item 力是使物体产生加速度的原因.加速度和力存在着
这样的关系:有力作用就有加速度产生,恒定的力产生恒定的
加速度,力发生变化,加速度随即发生相应的变化,力停止作
用,加速度立即等于零,而且不论在哪种情况下,加速度和产
生它的力的方向始终是一致的。
\item 物体运动的加速度不仅决定于外力,还同时跟物体
的质量有关,在这里,质量的大小对运动状态改变的快慢程
度起了一种制约作用,所以说质量是物体惯性大小的量度。
\item 把牛顿第二定律的实验结论$a\propto \dfrac{F}{m}$
改写成等式$F=kma$时,式中的比例常数$k$的取值跟式中其他物理量所用的
单位有关。在国际单位制中,质量$m$的单位是kg,加速度
$a$的单位是${\rm m/s^2}$, 力$F$的单位是N,于是:
\[k=\frac{F}{ma}=\frac{1{\rm N}}{1{\rm kg}\x 1{\rm m/s^2}}=1\]

这样,就可以使牛顿第二定律的公式简化为
\[F=ma\]
因此在应用牛顿第二定律公式解题时,必须使$m$、$a$、$F$三个
量都选用国际单位制单位。

\item 力的独立作用原理进一步阐明了加速度对力的依存
关系。不论物体是否还受有其他力的作用,也不论物体的初
始运动状态如何,一个力总是独立地使物体产生一个加速度。
譬如在落体运动或抛体运动中,不论是否存在空气阻力,重力
产生的加速度总是等于$g$. 也正因为这样,才能应用力的矢
量合成法则,把牛顿第二定律推广为$F_{\text{合}}=ma$, 力的独立作用
原理也为以后学习运动合成、波的叠加等知识准备了基础。
\end{enumerate}

\subsubsection{力、加速度、速度以及速度改变的关系}

加速度和力
有着直接的关系,但是物体运动的速度跟力并没有直接的关
系.学生常有如下的错误看法:
\begin{enumerate}
    \item 认为作用力大,则速度
一定大,作用力小,速度就不可能大;
\item 认为作用力如果
减小,则物体的速度也将逐渐减小。
\end{enumerate}
这两种错误都要进行
纠正。作用力大产生的加速度大,速度是否大还要取决于
初速度和加速的时间。这可以匀变速直线运动为例来加以
说明:即使加速度$a=F/m$
很大,但初速度$v_0$和加速时间$t$
如果很小,则$v_t=v_0+at$也不会很大;另一种情况,$a=F/m$
并
不大,但只要初速度$v_0$或加速的时间$t$很大,则$v_t$也不一
定小。后一种错误看法可以结合课本练习四第3题
进行讨论,指出:只要力$F$的方向和物体的初速度方向相
同,即使力$F$在逐渐减小,产生的加速度$a$也逐渐减小,但是,
物体的速度还是增的,只是每隔单位时间增加的速度$\Delta v$
在逐渐减小;直到当$F=0$, $a=0$时,物体的速度就不再增
大,物体将以做变速运动时最大的末速度为速度做匀速直线
运动。关于这道题目还可用一个比喻来加以说明:如果往银
行存钱,每个月存入的钱数在逐渐减少,但存款数还是在增
加的。

\subsubsection{关于质量和重量的区别与联系}

在物理学的传统讲
法中,质量和重量是两个不同的概念,质量就其本身的含义来
说,就是组成物体的物质多少;从动力学的观点来看,是物体
惯性大小的量度。质量是标量,物体的重量则是指物体所受到
的地球的重力,是矢量。

课文中先后出现的两个式子$G=mg$和
$\dfrac{G_1}{G_2}=\dfrac{m_1}{m_2}$
表明了质
量和重量的联系。教学时对这两个式子的意义可掌握以下的
分寸:
\begin{enumerate}
    \item $G=mg$这一关系式,在初中把比值$g$作为一个比例
常数,$g=9.8{\rm N/kg}$。它的意义是质量为1千克的物体的重
量是9.8牛顿.而高中课本则是从牛顿第二定律$F=ma$推导得出的,这是指物体在重力作用下运动时,重力产生的加
速度为$g$, 物体的质量为$m$, 则物体所受的重力$G=mg$. 两
处对$G=mg$的讲法,着眼点不同,但含义是一致的。

\item $\dfrac{G_1}{G_2}=\dfrac{m_1}{m_2}$一式所表明的意义是在地球上同一地点,物体的重量正比于物体的质量。这就是说,在地球上同一地点,
一切自由落体的加速度$g$都相等,在重力作用下,$g$的大小和
方向都是确定的。即比值$g=G/m$
是与物体的质量$m$无关的
一个量。

\item 由于现在已经规定重量作为质量的同意词,不再用
来表示物体所受的重力。因此,今后中学物理教学中如何处
理有关重量的问题,尚须进一步研究。
\end{enumerate}

\subsection{第三单元}
这一单元主要是综合应用牛顿运动定律和运动学的知识
来解决一些实际问题,并了解什么是超重和失重现象,解牛
顿定律的适用范围。

\subsubsection{牛顿运动定律的应用}

教材的安排是先通过应用
(一)讨论从物体的受力情况确定物体的运动情况;再通过应
用(二)从物体的运动情况确定物体的受力情况,要引导学生
认识在分析实际问题时,这两个方面常常是结合使用的,而对
运动物体受力情况的分析则是解决任何力学问题的基本出
发点。

\subsubsection{物体受力分析}

在对物体进行受力分析时,要让学生
注意首先应明确研究的对象,考察研究对象和周围哪些物体
有联系,并要结合物体的运动状态来分析。这是分析物体受
力情况时的一条基本思路,譬如在分析课本126页例题2
时,应向学生指出:题目是要求线对小球的拉力,因此小球应
是研究的对象;跟小球发生联系只有地球和悬线两个物体,也
就是小球只受到重力和悬线拉力这两个力的作用。既然小球
跟随小车一起做匀变速运动,则小球所受的合力必然不等于
零,而且小球运动的加速度必定是由重力和悬线的拉力的合
力所产生,又由于小球是沿着水平方向做匀变速直线运动,所
以这一合力也必定是沿着水平方向。于是就可以按课本图
3.11所示进行力的合成.由于合力所产生的加速度已由题
目给出,所以就可计算出悬线对小球的拉力。

\subsubsection{超重和失重} 在讲解超重和失重现象时,可以让全体
学生利用弹簧秤做课本图3.12的实验(弹簧秤可用
量程为250克力或400克力的,物体可用100克或200克的
钩码)。

在弹簧秤静止的情况下,观察指示器所指的读数就等于
钩码的重量。使弹簧秤急剧上升,就可以看到弹簧秤指示器
所指的读数突然增大,但随即又减小了,要指导学生观察开
始上升时弹簧秤读数增大这一变化。为了看清楚这一点,可
以让学生在弹簧秤挂了钩码静止的情况下,用一小块纸片卡
在弹簧秤指示器的下面,当弹簧秤加速上升时,指示器下降将
会把纸片推到下面读数较大的位置上。这样,当弹簧秤静止
时就可以看出弹簧秤读数曾经增大到多大了。

同样,在观察物体的失重现象时,应指导学生观察当弹簧
秤突然加速下降时读数减小的现象。为了观察读数曾经减小
到什么程度,也可用一小纸片卡在弹簧秤指示器的上端,用
以记录失重时的读数。

在分析超重和失重现象时,应指出:

\begin{enumerate}
    \item 挂在弹簧秤下的砝码只受重力和弹簧拉力的作用,
当钩码在竖直方向上做变速运动时,重力和拉力的合力不可
能等于零。这一合力就是使钩码产生加速度的力。在钩码向
上做加速运动或向下做减速运动时,所需的合力方向都是向
上的,而重力的方向是竖直向下的,因此只有依靠弹簧秤的拉
力大于重力,才能使得合力方向向上。这样,弹簧秤的读数就
会大于钩码的重量,这就是产生超重现象的原因,而当钩码
向下做加速运动或者向上做减速运动时,所需的合力的方向
向下,跟物体重力方向一致,因此重力中的一部分就用来产生
向下的加速度。由于重力“用掉”一部分,因此钩码作用于弹
簧秤的力将减小,即弹簧秤的读数将减小,这就是产生失重现
象的原因。
\item 如果物体向下做加速运动的加速度等于$g$, 则物体
的重力将全部用来产生加速度,这样就没有多余的力作用于
弹簧秤,弹簧秤的读数应等于零。

可以由教师做一演示,将挂有重物(钩码)的弹簧秤用线
悬在高处,在弹簧秤的指示器上贴一小块白胶布以便于观察,
这时弹簧秤的读数等于重物的重量。当用火柴烧断悬线后,
重物和弹簧秤一起自由下落,在这过程中可以观察到弹簧秤
的读数立即减小到零(演示时在落地处准备好一个网兜以承
接弹簧秤使之不会摔坏)。指出这是完全失重的现象。

\item 要使学生明确,不论超重还是失重现象,物体所受到
的重力是没有变化的,只是在物体静止时,支承这个物体的另
一物体(如桌面、悬线等)发生形变,产生了弹力作用于物体,
使物体保持平衡。当物体加速下降时,由于重力中的一部分用
来产生向下的加速度,物体需要的支持力变小了。相反,物体
加速上升时,物体所受的重力虽然也没有变化,但重力并不能
产生向上的加速度,因此支持物提供的弹力除了克服物体的
重力外还要包括使物体产生向上加速度所需的力,因而支持
物的弹力将增大,所以通常用弹簧秤(测力计)来测量物体的
重量时,必须使物体处于静止状态,避免失重或超重的影响。
\end{enumerate}

\subsubsection{牛顿运动定律的适用范围}

只需简单地向学生介绍,
牛顿定律只适用于解决低速运动问题,而不能适用于研究高
速运动,对于微观粒子的运动,一般说来,牛顿力学也是不
适用的。可指出所谓高速是指粒子的速率已高达可与光速相
比拟的程度。关于爱因斯坦狭义相对论的时空观,只需大致
地根据课本所叙述的内容进行介绍,不必作深入的讲解。这
节教材的意图在于一方面扩大学生的知识面,更重要的是使
学生认识到任何规律都有它的适用范围。

\section{实验指导}
\subsection{演示实验}

\subsubsection{阻力是使物体运动变慢的原因}
小车自斜面滑下后在水平面上运动的演示实验,可用宽
约12—15厘米、带有档边(高约0.8—1.0厘米)的两段木板,
分别作为斜面和水平面拼接起来,作为斜面的木板长约60厘
米,作为水平面的木板长约120厘米.在水平长木板的一端
装一个用扁铁作边框的网兜,如图3.1所示.长木板平面上
贴上塑料装饰板,在上面平铺一层软棉布(灯芯绒或绒布),棉
布上面再铺放两层毛巾布,以增大阻力。将作为斜面的木板
的一端用木块垫起,倾角$5^{\circ}$左右.要注意斜面跟长木板平面
的拼接处要尽可能吻合。
\begin{figure}[htp]
    \centering
    \includegraphics[scale=.6]{fig/3-1.png}
    \caption{}
\end{figure}

实验时将小车放在斜面顶端由静止开始释放。小车到达
铺有毛巾布的平面上后,运动30厘米左右即停下来;然后把
两层毛巾布取下(注意用薄木片调节拼接处的高度使之吻合),
再把小车放在斜面顶端由静止开始释放,小车在铺有软棉布
的平面上运动的距离将增大到60厘米左右.如果把棉布取
下,小车在斜面的同一高度滑下后,可以沿着塑料平面一直运
动下去直到落入网兜里。

\subsubsection{用气垫导轨观察骑块的运动}
使用前必须将导轨调节水平(沿导轨方向和垂直于导轨
方向都应用水平仪进行调节)。用软布擦去骑块内侧以及导
轨表面的浮灰。然后接通气泵,如发现气孔有堵塞现象,可用
细铜丝(从多芯软导线中抽出一根使用,不要用钢丝,因为钢
丝太硬会使气孔边缘卷口起毛,增大骑块运动时的阻力)通一
下,务必使每个气孔保持畅通。

实验时应先开动气泵,待正常供气后,再将骑块放上导
轨,轻轻推动一下,便可观察到骑块的运动很接近于匀速直线
运动。应强调指出,所要观察和研究的是骑块已经开始运动
以后的一段过程,由于摩擦阻力很小可以忽略,重力和气垫的
托力相互平衡,合力为零,因此骑块的匀速直线运动状态并不
发生改变,当骑块到达导轨一端被弹簧片弹回反向运动时,可
指出这时由于受到外力作用,迫使骑块改变了运动方向(也就
是改变了运动状态),但在反向运动的过程中,由于骑块所受
合力为零,骑块的匀速直线运动状态仍不发生改变。

\subsubsection{惯性现象}
如图3.2所示,将小车(质量约为200克)用细线通过定滑
轮和一质量较大的钩码(譬如30克)相连,用手挡住小车,不使
它运动,将一木块(体积约为$3\x6\x10{\rm cm^3}$)竖起来放在车上。
释放后,小车突然做加速运动,木块由于惯性将向后面倾倒。
\begin{figure}[htp]
    \centering
    \includegraphics[scale=.6]{fig/3-2.png}
    \caption{}
\end{figure}

利用同一装置继续演示,开始时用手扶住木块同时施加
适当的阻力使小车的加速度不至太大。待小车和木块一起运
动后再完全放手,这时木块和小车将保持相对静止,当小车
运动到平面一端受阻突然停止运动时,木块由于惯性将向前倾倒。



