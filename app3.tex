\chapter{附录一~~国际单位制(SI)}

我们知道,物理公式在确定物理量的数量关系的同时,也
确定了物理量的单位关系.因此,只要我们选定为数不多的
几个物理量的单位,就能够利用它们推导出其他物理量的单
位.这些被任意选定的物理量叫做\textbf{基本量},如力学中的长度、
质量和时间就是三个基本量,基本量的单位,如米、千克、秒
等,叫做基本单位,由基本量根据有关公式推导出来的其他
物理量,叫做\textbf{导出量},导出量的单位叫做\textbf{导出单位}.
\begin{table}[htp]
    \centering
    \caption{国际单位制的基本单位}
\begin{tabular}{cccc}
    \hline
    物理量名称 & 单位名称 & \multicolumn{2}{c}{单位符号}\\
&&中文 & 英文\\
\hline
长度 & 米 & 米 & m\\
质量 & 千克 & 千克 &kg\\
时间 & 秒 & 秒 &s\\
电流 & 安培 & 安 &A\\ 
热力学温度 & 开尔文 & 开 &K\\ 
发光强度 & 坎德拉 & 坎 & cd\\
物质的量 & 摩尔 & 摩 & mol\\
\hline
\end{tabular}
\end{table}

所谓单位制,就是有关基本单位、导出单位等一系列单位
的体制,由于所采用的基本量的不同,基本单位的不同,以及
用来推导导出单位的定义公式的不同,存在着多种单位制.多
种单位制并用,给科学技术的交流和发展带来不便,为了避
免多种单位制的并存,国际上制订了一种通用的适合一切计
量领域的单位制,叫做国际单位制,国际代号为SI.国际单
位制是1960年第十一届国际计量大会通过的,其后并向全世
界推荐使用,现在世界上许多国家采用了国际单位制或者正
在向国际单位制过渡,我国也统一实行以国际单位制为基础
的法定计量单位.

在力学范围内,国际单位制规定长度、质量和时间为三个
基本量,它们的单位用米、千克、秒为基本单位,对于象热学、
电磁学、光学等学科,除了上述三个基本单位外,还要加上另
外的基本量,并选定合适的基本单位,才能导出其他物理量的
单位.这样,国际单位制的基本单位共有七个.表1和表2
分别列出了国际单位制的基本单位和常用的力学量的国际单
位制单位.



\begin{table}\centering
    \caption{常用的力学量的国际单位制单位}
    \begin{tabular}{cc|cc|c}
      \hline
        \multicolumn{2}{c}{物理量}     & \multicolumn{2}{|c|}{单位}& 量纲式\\
名称 & 符号 & 名称 & 国际符号 &\\
\hline
面积    &  $S$    &  平方米    & ${\rm m}^2$ & $[L^2]$ \\
体积   &  $V$    &   立方米   &  ${\rm m}^3$ &$[L^3]$\\
位移   &  $s$    &   米   &  ${\rm m}$ &$[L]$\\
速度   &   $v$   &    米每秒  & ${\rm m}/{\rm s}$ &$[LT^{-1}]$  \\
加速度   &  $a$    & 米每二次方秒     &  ${\rm m}/{\rm s^2}$ & $[LT^{-2}]$\\
角速度   &  $\omega$    &  弧度每秒    & ${\rm rad}/{\rm s}$ & $[T^{-1}]$ \\
角加速度   &  $d$    &   弧度每二次方秒   &  ${\rm rad}/{\rm s^2}$ & $[T^{-2}]$\\
转速   &   $n$   &   1每秒   &  ${\rm s}^{-1}$ & $[T^{-1}]$\\
频率   &  $\nu,\; f$    &  赫兹    &  Hz & $[T^{-1}]$\\
密度   &  $\rho$    &  千克每立方米    & ${\rm kg}/{\rm m^3}$  &$[L^{-3}M]$ \\
力   &  $F$    &  牛顿    &  N &$[LMT^{-2}]$\\
重量   &  $G$    &    牛顿  & N  &$[LMT^{-2}]$\\
力矩   &  $M$    &    牛顿米  & ${\rm N}\cdot {\rm m}$  & $[L^2MT^{-2}]$\\
动量   &  $p$    & 千克米每秒     &  ${\rm kg}\cdot {\ms}$ &$[LMT^{-1}]$\\
冲量   &  $I$    &  牛顿秒    &   ${\rm N}\cdot {\rm s}$&$[LMT^{-1}]$\\
压强   &  $p$    &   帕斯卡   &  Pa & $[L^{-1}MT^{-2}]$\\
功   &  $W$    &   焦耳   &  J & $[L^2MT^{-2}]$\\
能   &   $E$   &   焦耳   &  J & $[L^2MT^{-2}]$\\
功率   &  $P$    &  瓦特    & W &$[L^2MT^{-3}]$ \\
\hline
    \end{tabular}
\end{table}	





\chapter{附录二~~量纲}

我们知道,物理量可分为基本量和导出量,既然导出量
可以从基本量导出,那么每个导出量一定可以用基本量的某
种组合表示出来.表示一个物理量是由哪些基本量组成和怎
样组成的式子,叫做这个物理量的量纲式,在国际单位制中,
所有的力学量都是由长度、质量、时间这三个基本量组成的,
如果用$L,M,T$分别表示这三个基本量,那么,一个物理量$Q$
的量纲式的一般形式就是
\[[Q]=\qty[L^{\alpha}M^{\beta}T^{\gamma}]\]
其中$\alpha, \beta, \gamma$分别叫做物理量$Q$对于长度、质量、时间的\textbf{量纲}.

下面举出几个物理量的量纲式:
\begin{itemize}
    \item 体积的量纲式
    \[[v]=\qty[L^3]  \]
    \item 速度的量纲式
    \[[v]=\frac{[s]}{[t]}=\qty[LT^{-1}]  \] 
    \item 加速度的量纲式
    \[ [a]=\frac{[v]}{[t]}=\qty[LT^{-2}] \]
    \item 力的量纲式
    \[ [F]=[m][a]=\qty[LMT^{-2}] \]
\end{itemize}

在上面的表2中列出了一些力学量的量纲式.

物理量的量纲式可以用来检验物理关系式的正确性.检
验时所根据的原则是:只有量纲式相同的项才能相加减,而且
等号两边一定要有相同的量纲式.一个正确的物理关系式,
一定是符合上述原则的.



















